\subsection{Home Automation and Smart Homes}
Home automation is a growing industry more and more homes get more and more automation. But it is still an unexplored technology and are far from every being in every home. Though the technology is progressing, Home Automation have been quite a long time under way compared with nearly all other electronics and IT technologies. It was first introduced in the 1930s during many of the fairs around the world, especially the 1933 Homes of Tomorrow Exhibition at the A Century of Progress International Exposition in 1933 to 1934 in Chicago. So even with over 80 years in the business it is still a rarity to own, or for that matter even knowing someone that own, a Home Automation system. Compared to mobile phones, first commonly theorised in 1948 in the science fiction novel Space Cadet by Robert Heinlein, and now with modern society practically demanding people to own one that is at maximum 2-3 years old, Home Automation seems to be lagging behind. So how come this is the case? How come this technology is so hard to popularize. To find out we first look at the state of the art solutions to find out what they do and what could be missing. Many systems exists that could be looked at but the following were chosen due to their representation of the different aspects and relative high popularity.

\subsubsection{Home Seer}
%source on first statement
Home Seer is one of the highest rated systems on the market. It is a modular home automation system where the users are able to control and monitor their homes via an application on their smart phones. The user buys a controller that acts as the central computer system and can after that buy a very large range of modules of different types spanning over multiple brands depending on which controller they own. The system is split into 17 main categories.\footnote{The categories will not be explained in detail here but will be used and explained more in detail in the design if they are used.}
\begin{itemize}
	\item Lighting
	\item Thermostats / Climate Control
	\item Door Locks
	\item Garage Doors
	\item Cameras
	\item Security Systems
	\item Appliances
	\item Sensors
	\item Water Management
	\item Shades / Blinds
	\item Voice, Telephony
	\item Audio / Video / Media
	\item Energy Management
	\item Weather
	\item Automobile
	\item Fitness / Wearables
	\item Pool / SPA Control
\end{itemize}%Source: http://www.homeseer.com/compatible-products.html
When the devices are connected to the controller the system can be controlled by any computer on the local network using the special designed software HS3/HS3PRO. In this software the user is able to manage the controller to set-up, manage, and remove hardware in the system. What makes this system somewhat unique is that the system allows the user to program each of the units individually. Allowing the user to set-up automated events like turning on lights at sunset or locking the door and closing the garage door when the car leaves the driveway. The system can also be set up to be controlled and monitored via an application on their mobile device like seeing which ligths are turned on and being able to turn them off remotely.%http://www.homeseer.com/guides/HomeSeer-QuickStart-Guide.pdf

\subsubsection{Control4}
Control4 is in many cases the more user-friendly version of Home Seer. An expert is assigned to the user and they together work out what the users needs are and how they want their home to be automated. The system is a lot less modular but is customised specifically to the people using it, both in software and hardware. This also makes it possible for Control 4 to focuses more on the home \enquote{knowing} when to do something, and actively doing it instead of relying on inputs from the user like an app or similar. This also means the system will make less mistakes and give the user a perceived higher quality product but at a much higher cost.%http://www.control4.com/

\subsubsection{Samsung: SmartThings}
Samsung: SmartThings is effectively a middle-ground between Home Seer and Control4. It tries to make a user friendly and partially customisable smart home but at a relatively low price. This is done using common patterns for people, like coffee brewing in the morning and washing of clothes before returning home. The set-up is short and only consists of a few questions for the user to personalise the experience to some degree. This is not as personalised as Control4 and Home Seer can be and is limited by the use cases designed by Samsung, but is a lot cheaper then Control4 and a lot easier and more user friendly than Home Seer.

The system works as a conversation between the user and the system. Making the user trigger certain events like morning routines by sending the message \enquote{Goodmorning} and reporting with useful information and status' like \enquote{The weather forecast for today is \dots Setting the temperature to 24\degree and starting to brew your coffee.}

\subsubsection{Apple: HomeKit}
HomeKit is a framework for communicating with and controlling connected accessories in a user's home. %Directly copied from https://developer.apple.com/homekit/
This system is designed for superusers and programmers to use Apples voice activation and recogniser software Siri to be able to control their Home Automation system via voice commands.