%mainfile: ../../master.tex
\subsection{Areas of Home Automation}
\label{sec:Areas of Home Automation}
There is a certain amount of tasks, which have to be done on a daily basis in a typical household. These could be trivial tasks, such as turning off the light, when leaving the bathroom, or more advanced ones, such as cooking dinner or doing laundry. Depending on what type of task, it may be desirable to automate it, in order to increase the quality of life.
\\\\
In this section, some of the major areas which are usually considered in home automation systems will be discussed.
The major areas are the following:
\begin{enumerate}
  \item Security
  \item Surveillance
  \item Light automation
  \item Climate control
\end{enumerate}
These will be expanded on below.

\subsubsection{Security}
\label{sub:Security}
A specific area of security in home automation could be detection of fire, turning off the oven after it is done. Detecting a fire at the right moment is crucial as it could save either life or property. Monitoring and reporting hazardous events to a central in order to receive help in sufficient time automatically will mean there will be less to worry about and thus will improve on the quality of life.

\subsubsection{Surveillance}
\label{sub:Surveillance}
Automation in this area boosts the security. This type of automation is concerned with making the owner able to keep track of, for example, who is at the door and thereby giving the user the choice of opening the door remotely.

\subsubsection{Lighting Automation}
\label{sub:Lighting Automation}
Keeping track of lights in a home is a very typical task which has to be done daily. Automation in this area will not only increase the comfort of the user, but also help the environment by reducing energy consumption and increasing efficiency. The system could detect the amount of light as well as monitoring sunrise and sunset and thereby turning off the light, when it is not needed and thereby save energy.

\subsubsection{Climate Control}
\label{sub:Clitemate Control}
A trivial task that is maintained on a daily basis is indoor heat control; cooling the house down when it gets too hot and heating it up again when it gets too cold. Automation of heat control inside the home can contribute to solving environmental issues by automatically and intelligently monitoring and regulating the room temperature. By monitoring environmental influences, such as the weather and the user's preferred temperature, this task could be optimised to save energy and improve the life quality of the user.

\subsubsection{Delimitation}
Many different areas of the home can be automated by a home automation system. However, if a system is to be self-learning, there needs to be room for the system to make mistakes; thus some critical areas, such as security, should not be automated by such a system. Some of these areas also fit better with the idea of saving energy; lighting and room temperature regulation automation are perhaps the areas which have the biggest potential for saving energy. Therefore these areas are chosen as the main focus of this project.
