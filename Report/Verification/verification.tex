%mainfile: ../master.tex
\chapter{Test and Verification}
This section contains a description of the testing, performed on the system to verify that it, to a satisfying degree, solves the problem solution. Along with this description, a number of alternatives will be discussed in relation to why they were not chosen.

\section{Testing}
To test the system, a simple test system was set up in a controlled environment, consisting of just one sensor, an emulatable action, and the server. Then for *\ranote{insert how many hours we actually performed it} we performed a deliberate pattern. A pattern here, consisting of an action and a change in the sensor state. Then the test conductor stopped performing the action, and if the system took over and performed the action, when the sensor state changed, it would be considered successful. This is the simplest use case for the system, but shows that the learning procedure in the learning subsystem can recognise a pattern, which is also the smallest requirement for the system to be deemed working. Once the learning subsystem had been tested, the real-time critical part of the system (from the embedded subsystem through the reasoner back to the embedded subsystem) was tested.
\\\\
Once it has been verified that the system could learn a pattern and could reason and subsequently perform the appropriate action within 100 ms, as specified in \cref{sub:people}, the second key component of the system, the ability to adapt to a change in patterns, was tested. So, following the first test, a different overlapping pattern was performed for an hour. Then it was checked if the system had adapted, if not, the pattern was performed for another hour. This was repeated until the system adapted. While it is important to note how quickly the system adapts to change, once it has been confirmed that the system does adapt, the speed with which it does can be vary.
\subsection{Results and Specific Setup}
All the logs can be found in \cref{app:cd}. \ranote{remember to put the logs on the CD}
\subsubsection{Single pattern test}
The setup consisted of a single sensor station with two switches. The switches were connected to  Light-Emitting Diodes (LED) such that each switch controlled one LED. The pattern being performed; when turning on one LED the other one is turned on as well. A snippet of the samples this produced can be seen in \cref{Table:SampleSnippet}, where time moves from right to left so the rightmost state is the newest.
\begin{center}

\begin{table}[htbp]
  \centering
  \begin{tabular}{c c c c c c}
    \toprule
    Sample & & & & &  \\ \midrule
    1 & 1 & 3 & 2 & 0 & 1 \\
    2 & 3 & 2 & 0 & 1 & 3 \\
    3 & 2 & 0 & 1 & 3 & 2 \\
    4 & 0 & 1 & 3 & 2 & 0 \\
    5 & 1 & 3 & 2 & 0 & 1 \\
    6 & 3 & 2 & 0 & 1 & 3 \\
    7 & 2 & 0 & 1 & 3 & 2 \\
    8 & 0 & 1 & 3 & 2 & 0 \\
    9 & 1 & 3 & 2 & 0 & 1 \\
     \\ \bottomrule
  \end{tabular}
  \caption{Samples from test with simple pattern, as performed by the user}
\end{table}

\label{Table:SampleSnippet}
\end{center}
Once 31 samples had been received, as can be seen in a snippet from the learner log\cref{Listing:MarkovGenLog}, the learner were run and a model generated.
\lstset{language=xml}
\begin{lstlisting}[label = Listing:MarkovGenLog, caption = Snippet of log from model generation]
  <date>2015-12-17T09:03:32</date>
  <logger>aiLogger</logger>
  <message>Sample size for generating hidden markov model: 34</message>
\end{lstlisting}
With this model the system started trying to predict actions. At this point no user actions were being performed on sensor id 1. The confidence threshold for this test were 75\%. The log trace for the first action can be seen in \cref{Listing:CompletActionTrace}. And as can be seen the in the logs the system predicted the state at 09:03:36 based on the state from 09:03:33. Looking at the logs beforehand that shows user actions, it can be seen that this pattern was repeated by the user.
\begin{lstlisting}[label = Listing:CompletActionTrace, caption = Snippets from different logs to show how the process of making an action]
	<record>
	  <date>2015-12-17T09:03:33</date>
	  <millis>1450339413544</millis>
	  <logger>sampleLogger</logger>
	  <thread>1</thread>
	  <message>Sample:  3 2 0 1 3</message>
	</record>
	
	<record>
	  <date>2015-12-17T09:03:33</date>
	  <millis>1450339413553</millis>
	  <logger>aiLogger</logger>
	  <thread>1</thread>
	  <message>Confidence: 0.8668783888317774. Actions: 
	Set sensor id 1 to value 0</message>
	</record>
	
	<record>
	  <date>2015-12-17T09:03:33</date>
	  <millis>1450339413555</millis>
	  <logger>reasonLogger</logger>
	  <thread>1</thread>
	  <message>Sending data: Set sensor id 1 to value 0</message>
	</record>
	
	<record>
	  <date>2015-12-17T09:03:36</date>
	  <millis>1450339416153</millis>
	  <logger>sampleLogger</logger>
	  <message>Sample:  2 0 1 3 2</message>
	</record>
\end{lstlisting}
Based on this it can be concluded that the system successfully learned and imitated a pattern.

\subsubsection{Real Time Test}
To perform this test the system were first taught a simple pattern as it were important, that a certain action on the arduino forced a prediction of an action by the server model. A separate button were added that had no other function then to start the test (ie. it were not part of the pattern). This allowed the test code to artificially create an action on the other button and then immediately start a timer. The button were then disabled as to avoid restarting the timer. Then once the arduino received the action the timer were stopped and the time for a full loop of the system were found. One thing to note, this time can vary depending on when in the arduino loop the button were pressed. To document this two tests were performed one were the artificial button press happened at the theoretically best time and one at the theoretically worst time. The worst case happens when we perform the button press after we send a state to the server but before we start receiving data. This means we have to wait for the receiving delay, then send again, and then wait for the receive delay again. For the best case test, the lowest observed time were 140ms\ranote{insert actual test}, which is 40ms above the deadline. This could be attributed to the send and receive of the xbee modules which in practical testing showed a delay of up to 60ms for one send, however theoretically the worst case should be 50ms\cref{xbee_latency}. This variance could be due to a number of things, eg. cheap hardware or non-optimally implementation. For each other separate module we have the approximate time, and so an average case for a full loop of the system would be 155 as seen in \cref{Table:RunTimeAprox}.
\begin{center}
	\begin{table}[htbp]
	  \centering
	  \begin{tabular}{c | c}
		\toprule
		Sample  		& 			\\ \midrule
		Test Start		&
		encode 			& <2ms  	\\
		send   			& <50  		\\
		Reason loop 	& avg: 53 	\\
		Receive 		& <50  		\\
		Total			& 155		\\
									\\ \bottomrule
	  \end{tabular}
	  \caption{An approximation of run time of each separate module, from single pattern test. When performing an action at the optimal time in the arduino loop}
	\end{table}
 \label{Table:RunTimeAprox}
\end{center}
The test for the worst case test, were .... best case and worst case.
The average case for this test can be seen in \creft{Table:WorstRunTimeAprox}

\begin{center}
	\begin{table}[htbp]
	  \centering
	  \begin{tabular}{c | c}
		\toprule
		Sample  		& 			\\ \midrule
		encode 			& 		  	\\
		Test Start		&			\\
		send   			& <50  		\\
		Reason loop 	& avg: 53 	\\
		Receive 		& <50  		\\
		encode 			& <2ms  	\\
		send   			& <50  		\\
		Reason loop 	& avg: 53 	\\
		Receive 		& <50  		\\
		Total			& 308		\\
									\\ \bottomrule
	  \end{tabular}
	  \caption{An approximation of run time of each separate module, from single pattern test. When performing an action at the worst time in the arduino loop}
	\end{table}
 \label{Table:WorstRunTimeAprox}
\end{center}










\section{Alternative tests}
The first alternative which were considered is a real world use case. In this test a system would be setup in one or more users homes for a given set of time. Once the test had finished the log data from the system along with user testimony would determine the performance of the system. This test is time consuming, because the system would need to be live for an extended period of time to learn user patterns on a daily, weekly, or longer scale. The log data can also be difficult and time consuming to analyse because the large amount of data processed by the system and the difficult to visually represent machine learning model used.
\\\\
Unit tests is another alternative which ensures some stability in the system. But for a system that relies on random data with patterns, which is difficult to produce programmatically, unit tests does not show how well the system performs only that it produces some data. A preliminary unit test were performed on the system. In this test the system were given a specific set of sensor states a set number of times and then checked whether the system produced the expected action given a sensor state.
