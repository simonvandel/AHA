\section{Hardware}
This section discusses hardware that is used to accomplish a desired solution.  The hardware is described, tested, and reviewed. A brief description of the hardware is included in this section followed by a review of the hardware.

\subsection{PIR Motion Sensor}

The sensor used to detect movement is a passive infrared sensor (PIR). The model
number is \enquote{SEN32357P} and is manufactured by SEEED Studio. A data sheet
can be found on:
\cite{datasheet_pir1}. The
sensor is implemented using the \enquote{BISS0001} IC. A datasheet for this can
be found on \cite{datasheet_pir2}.

According to the technical specifications, the sensor can measure movements from 0.1 m to 6
m away. The distance can be configured by rotating a potentiometer on the
sensor. Clockwise means decreasing the distance. This is essentially the
sensitivity of the sensor. A lesser distance means lower sensitivity. The detecting angle is 120 degree.

The sensor also has a potentiometer for configuring the time delay. The time
delay is the time the sensor reports movement after a movement is detected. So
if a movement is detected, the sensor with time delay 10 seconds is guaranteed
to report motion detected for 10 seconds. The time delay on this specific sensor
can be adjusted from 1 second to 25 seconds. A switch on the sensor controls
whether the sensor is retriggerable (H position) or unretriggerable (L
position). In a retriggerable position the delay time is extended every time
movement is detected. In the unretriggerable mode, the delay time remaining is
not reset when motion is detected.

The sensor has 4 pins, GND, VCC, NC and SIG. The sensor signals motion detected
on its SIG pin. Low on this pin means no motion and HIGH means motion.

An example of a setup can be seen in \cref{fig:arduino_pir_wiring}.

\begin{figure}[htbp]
  \centering
  \includegraphics[width=\textwidth]{arduino-pir-wiring.png}
  \caption{The figure depicts wiring for a PIR motion sensor. A LED is shown in
    the figure, for a lack of a PIR component in the software generating the
    wiring schematics.}
  \label{fig:arduino_pir_wiring}
\end{figure}

\subsubsection{Sampling Input Data}

To reduce noise on the signal of the sensor, a simple statistical algorithm is
employed on the input data. Essentially the data is reduced to two tiers
of data. First, the input data is accumulated into a tier1 low and a tier1 high
counter. When $n_{tier1}$ number of samples have been colliwcted, a tier2 sampling round begins. If tier1
had more high counts than low counts, tier2 high is incremented and the opposite if
tier1 had the most lows. When the sample count of tier2 has reached a number
$n_{tier2}$, the calculated signal can be reported. If tier2 had the most highs,
report high and vice verca for tier2 lows.

\sinote{This is a very simple algorithm, so surely a better exists?}

\subsection{Experiments}
\label{sub:Experiments}
In order to determine the reliability of the hardware several experiments are conducted with the PIR sensor.
The tests are all based on data found in the datasheet for the hardware.
The main purpose of the tests is to determine the correlation between real world performance and the hardware specifications
found on the datasheet.
\\\\
Four tests are carried out with one main setup that are applied to all of them. The four consists of the following:

\begin{enumerate}
  \item Sensitivity Test
  \item Angle Test
  \item Distance Test
  \item Delay Time Test
\end{enumerate}
\subsubsection{The Setup}
\label{subs:The Setup}
As aforementioned the setup for all the experiments are identical
that is to ensure consistency of the results by reducing variables, that affects
the results, as much as possible.  The PIR sensor is placed in place surrounded wall
and where other moving objects are limited to a certain degree. In order to ensure that the
results are as precise as possible, all four tests are carried out at the same exact place but
the method of the individual test differs.
The PIR sensor is placed in a certain height and made sure the sensor itself is
sitting still during the tests.
\paragraph{Sensitivity}
\label{par:Sensitivity}

\subparagraph{Hypothesis}
\label{subp:SenHypothesis}
According to the datasheet the PIR sensor will be too sensitive if the distance potentiometer is rotated counter-clockwise.
The sensor should get so sensitivethat the PIR sensor would be triggered by the atmosphere even
if no moving object is existing\cite{datasheet_pir1}.
\subparagraph{Test Procedure}
\label{subp:SenTest Procedure}
The PIR sensor is configured with the distance potentiometer rotated
counter-clockwise as far as possible and the first triggered to test that everything works.
After the sensor is configured every object that can affect the sensor is removed and the sensor is observed.
The observation is to determine whether the sensor will detect objects are it is not
supposed to detect and how sensitive the sensor really can is.
\subparagraph{Results}
\label{subp:SenResults}

\subparagraph{Partial Conclusion}
\label{subp:SenPartial Conclusion}


\paragraph{Angle}
\label{par:Angle}

\subparagraph{Hypothesis}
\label{subp:AngHypothesis}
According to the datasheet of the PIR sensor,
the sensor has a detecting angle of 120 degrees.
The sensor should according to the datasheet not detect any angle above 120 degrees.

\subparagraph{Test Procedure}
\label{subp:AngTest Procedure}
The sensor is placed in a certain position and detecting angle is measured equivalent to
the specified angle found in the datasheet.
For the sake of explanation let the angle above 120 degree be the grey zone and
the detecting area that is within the 120 degree white zone.
A moving object is placed in the grey zone and slowing moving into the white zone.
The PIR sensor should be triggered when the moving object enters the white zone in order
to work exactly as specified in the hardware specifications.
The zone of the moving object and PIR sensor is observed accordingly
and thereby the actual detecting angle is determined.
\subparagraph{Results}
\label{subp:AngResults}

\subparagraph{Partial Conclusion}
\label{subp:AngPartial Conclusion}

\paragraph{Distance}

\subparagraph{Hypothesis}

According to the datasheet, the sensor can measure motion from 10 centimeters to
6 meters.

\subparagraph{Test procedure}

The sensitivity setting of the sensor essentially controls the distance. First,
the sensitivity is set to the lowest sensitivity. At that sensitivity, the
sensor should detect motion at a range of 10 centimeters, according to the above
hypothesis.

The sensor is measured by the test conductor approaching the sensor while
waving. Once the sensor detects the motion, the distance from the test conductor
to the sensor is measured. This is done three times to check for consistency. The sensitivity is then upped one unit on the sensor,
and the test conductor measures the distance for the new sensitivity.

\subparagraph{Results}

\subparagraph{Partial conclusion}


\paragraph{Delay time}

\subparagraph{Hypothesis}

According to the datasheet, the sensor has a delay time of 1 second to 25 seconds.

\subparagraph{Test procedure}

In this experiment, the sensor is set in the non-retriggerable setting (L
position), so the delay time is not extended on additional motion.

First the sensor is adjusted to have the maximum delay time of 25 seconds
according to the above hypothesis. This is then tested by triggering the sensor
and starting a stopwatch simultaneously. When the sensor stops signalling, the
stopwatch is stopped and the actual delay time noted. This is done three times
for each setting of delay time to check for consistency.

The next iteration of the experiment, the delay time potentiometer on the sensor
is rotated one unit, decreasing the value.

\subparagraph{Results}

\subparagraph{Partial conclusion}
