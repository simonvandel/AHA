\section{Hidden Markov Model}
The only information available for the machine intelligence model of the project are the sensors of the problem domain and time. For this we will use the term -\emph{snapshot}. A snapshot is here meant as the values of all sensors in the home at the same time, snapshot will from here on adhere to this definition. 

To model the machine intelligence the Markov model was chosen. This model en-copes discrete time changes in the states of a world. This is an nearly exact match to this project domain. Further more the model assumes a Newtonian world view. This means what is going to happen in the future is only dependent on what is currently true en the present, not what has happened in the past. We will now introduce the diagrams mainly used to illustrate how the models changes over time graphically. These diagrams are called Trellis diagrams. The simplest form of a Markov model can be seen in \cref{fig:1stMarkovModel}. Here we can see the property that at any state in time $s_t$ the next discrete time future $s_{t+1}$ is only dependent on the current state $s_t$.

\begin{figure}[htbp]
\centering
\begin{tikzpicture}
  %World States Nodes
  \node          (dots) [draw=none,  minimum width=1.25cm, minimum height=.75cm] {\LARGE \dots};
  \node[ellipse] (htp2) [draw=black, minimum width=1.25cm, minimum height=.75cm, right=of dots] {$s_{t-2}$};
  \node[ellipse] (htp1) [draw=black, minimum width=1.25cm, minimum height=.75cm, right=of htp2] {$s_{t-1}$};
  \node[ellipse] (ht)   [draw=black, minimum width=1.25cm, minimum height=.75cm, right=of htp1] {$s_t$};
  \node[ellipse] (htf1) [draw=black, minimum width=1.25cm, minimum height=.75cm, right=of ht] {$s_{t+1}$};
  
  %1. order transition lines
  \draw [->, to path={-| (\tikztotarget)}] (dots) edge[out=0,in=180] (htp2);
  \draw [->, to path={-| (\tikztotarget)}] (htp2) edge[out=0,in=180] (htp1);
  \draw [->, to path={-| (\tikztotarget)}] (htp1) edge[out=0,in=180] (ht);
  \draw [->, to path={-| (\tikztotarget)}] (ht) edge[out=0,in=180] (htf1);
\end{tikzpicture}
\caption[Trellis diagram for a simple Markov model]{The trellis diagram for a hidden Markov model with discrete time world state changes}\label{fig:1stMarkovModel}
\end{figure}
Even though these diagrams are good at illustrating the models and how they change over time it is important to remember that these are not illustrating the exact models themselves, only how they change over time for a specific sequence of observations. The actual model could be illustrated via causal/Bayesian networks, which also will be used later.

Even though the Markov model is this is a good fit, problems arises when using it. These problem arises since the model has a Newtonian world view. One could argue that this world view is also applicable for our problem domain; what the user is going to do is only dependent on the current physical state of the world and what the user is current thinking and remembering. The problem is that we can not sensor everything about the current complete physical state, or at least not everything affecting his/her decision, and differently not what the user is thinking and remembering.

We start by trying to encompass the current complete physical state that effect the users routines in their homes. The problem here is that this is individual for different users and to observe this, one would need enough sensors to encompass all individual peoples needs. This is very hard to find, even if everyone could be asked or represented, and everyone would need to since the project do not want to focus on a specific group of people, people do not necessarily know what factors affect their choices. Furthermore new and prior unknown use patterns could emerge after the implementation either because of the system or something externally or future users could simply not yet borne. Some sensors needed could simply not exist or be too imprecise or impractical. In short, finding all sensors needed to being able to observe all data needed for everyone’s use pattern is impossible in practice. To encompass this the we now introduce the hidden Markov model (HMM). The hidden states of the HMM are introduced to represent the exact reality, that cannot be observed. A hidden state can then emit an observable state. In this project this means that the hidden states are the true state of the world, all information needed for the user to take a choice on what to do next. This unobservable world then emits values to the sensors that we can observe, creating the observable states. Now the problem is however that these hidden states are all possible complete worlds and would in practice be infinitely many. The solution for this we find in the second problem. To illustrate the HMM we look to \cref{fig:2ndMarkovModel}. Here we can see that any given hidden state emits an observable state at any given time. 


\begin{figure}[htbp]
\centering
\begin{tikzpicture}
  %Hidden Nodes
  \node          (dots) [draw=none,  minimum width=1.25cm, minimum height=.75cm] {\LARGE \dots};
  \node[ellipse] (htp2) [draw=black, minimum width=1.25cm, minimum height=.75cm, right=of dots] {$x_{t-2}$};
  \node[ellipse] (htp1) [draw=black, minimum width=1.25cm, minimum height=.75cm, right=of htp2] {$x_{t-1}$};
  \node[ellipse] (ht)   [draw=black, minimum width=1.25cm, minimum height=.75cm, right=of htp1] {$x_t$};
  \node[ellipse] (htf1) [draw=black, minimum width=1.25cm, minimum height=.75cm, right=of ht] {$x_{t+1}$};

  %Emission states
  \node[ellipse] (etp1) [draw=black, minimum width=1.25cm, minimum height=.75cm, below=of htp1] {$e_{t-1}$};
  \node[ellipse] (etp2) [draw=black, minimum width=1.25cm, minimum height=.75cm, below=of htp2] {$e_{t-1}$};
  \node[ellipse] (et)   [draw=black, minimum width=1.25cm, minimum height=.75cm, below=of ht] {$e_t$};
  \node[ellipse] (etf1) [draw=black, minimum width=1.25cm, minimum height=.75cm, below=of htf1] {$e_{t+1}$};

  %Separator
  \coordinate (MW) at (0,-0.8);
  \node (HW) [above left=0ex of MW] {Hidden world $\uparrow$};
  \node (VW) [below left=0ex of MW] {Observable world $\downarrow$};
  \draw[dashed, draw=black] (-1,-0.8) -- (10.5,-0.8);
  %\draw[dashed] ($(htp2)!0.5!(etp2)$) -- ($(htf1)!0.5!(etf1)$);
  
  %Emission lines
  \draw[->, to path={-| (\tikztotarget)}] (htp1) edge[out=-90,in=90] (etp1);
  \draw[->, to path={-| (\tikztotarget)}] (htp2) edge[out=-90,in=90] (etp2);
  \draw[->, to path={-| (\tikztotarget)}] (ht) edge[out=-90,in=90] (et);
  \draw[->, to path={-| (\tikztotarget)}] (htf1) edge[out=-90,in=90] (etf1);
  
  %1. order lines
  \draw[->, to path={-| (\tikztotarget)}] (dots) edge[out=0,in=180] (htp2);
  \draw[->, to path={-| (\tikztotarget)}] (htp2) edge[out=0,in=180] (htp1);
  \draw[->, to path={-| (\tikztotarget)}] (htp1) edge[out=0,in=180] (ht);
  \draw[->, to path={-| (\tikztotarget)}] (ht) edge[out=0,in=180] (htf1);
\end{tikzpicture}
\caption[Trellis diagram for a simple Hidden Markov Model]{The trellis diagram for a simple hidden Markov model that implements a Newtonian worldview}\label{fig:2ndMarkovModel}
\end{figure}

To encompass what the user is currently thinking and remembering, essentially depend on what happened in the past and what is currently true. We therefore need a way to en-cope the past in the current state. This is called the order of HMM. The order of a HMM directly corresponds to how much any state $s_t$ is dependent on. This is better illustrated, see \cref{fig:3rdMarkovModel}.


\begin{figure}[htbp]
\centering
\begin{tikzpicture}
  %Hidden Nodes
  \node          (dots) [draw=none,  minimum width=1cm, minimum height=.75cm] {\LARGE \dots};
  \node[ellipse] (htp2) [draw=black, minimum width=1cm, minimum height=.75cm, right=of dots] {$x_{t-2}$};
  \node[ellipse] (htp1) [draw=black, minimum width=1cm, minimum height=.75cm, right=of htp2] {$x_{t-1}$};
  \node[ellipse] (ht)   [draw=black, minimum width=1cm, minimum height=.75cm, right=of htp1] {$x_t$};
  \node[ellipse] (htf1) [draw=black, minimum width=1cm, minimum height=.75cm, right=of ht] {$x_{t+1}$};

  %Emission states  
  \node[ellipse] (etp2) [draw=black, minimum width=1cm, minimum height=.75cm, below=of htp2] {$e_{t-2}$};
  \node[ellipse] (etp1) [draw=black, minimum width=1cm, minimum height=.75cm, below=of htp1] {$e_{t-1}$};
  \node[ellipse] (et)   [draw=black, minimum width=1cm, minimum height=.75cm, below=of ht] {$e_t$};
  \node[ellipse] (etf1) [draw=black, minimum width=1cm, minimum height=.75cm, below=of htf1] {$e_{t+1}$};
  
  %Emission lines
  \draw [->, to path={-| (\tikztotarget)}] (htp2) edge[out=-90,in=90] (etp2);
  \draw [->, to path={-| (\tikztotarget)}] (htp1) edge[out=-90,in=90] (etp1);
  \draw [->, to path={-| (\tikztotarget)}] (ht) edge[out=-90,in=90] (et);
  \draw [->, to path={-| (\tikztotarget)}] (htf1) edge[out=-90,in=90] (etf1);
  
  %1. order lines
  \draw [->, to path={-| (\tikztotarget)}] (dots) edge[out=0,in=180] (htp2);
  \draw [->, to path={-| (\tikztotarget)}] (htp2) edge[out=0,in=180] (htp1);
  \draw [->, to path={-| (\tikztotarget)}] (htp1) edge[out=0,in=180] (ht);
  \draw [->, to path={-| (\tikztotarget)}] (ht) edge[out=0,in=180] (htf1);
  
  %2. order lines
  \draw [->, to path={-| (\tikztotarget)}] (dots) edge[out=45,in=135] (htp2);
  \draw [->, to path={-| (\tikztotarget)}] (dots) edge[out=45,in=135] (htp1);
  \draw [->, to path={-| (\tikztotarget)}] (htp2) edge[out=45,in=135] (ht);
  \draw [->, to path={-| (\tikztotarget)}] (htp1) edge[out=45,in=135] (htf1);
\end{tikzpicture}
\caption[Trellis diagram for a 2nd order Hidden Markov Model]{A trellis diagram for a 2nd order hidden Markov model.}\label{fig:3rdMarkovModel}
\end{figure}

\Cref{fig:3rdMarkovModel} shows the trellis diagram for a 2nd order hidden Markov model. Here we can see that every state $s_t$ is dependent on two states in the past $s_{t-1}$ and $s_{t-2}$ meaning that every state is not only dependent on what happened the time step before. This can trivially be expanded to any nth order. Every model we have look at so far has been a 1st order Markov model. Any nth order Markov model where $n>1$ violates the Newtonian world-view, also known as the Markov property. Many algorithms within this field assumes this Markov property, including the ones later used in this project. Any nth order Markov model can be converted to a 1st order Markov model, thereby keeping the Markov property. This is in this project done by encoding the needed past, in form of the prior hidden states, in the hidden state together with the current.

On the basis of this we can now find the specific hidden states. We introduce the term \emph{sample} that covers the current observable state together with the history of snapshots n steps back. The size of this n will be discussed in \cref{}\jenote{need reference to section about the sampler}. This sample is used to find hidden states. A list of samples are created from an observation sequence of snapshots, further described in \cref{}\jenote{need sampler reference}. One could argue that this sample needed to be created from history of hidden states, not snapshots, since it is these we depend on in the current state. The problem here arises because both learning probabilities and the what nodes to include cannot really be done completely in the same graph. A compromise was therefore made so that the samples and thereby hidden states are found from the observations and history of observation, on the reasoning that we can not distinct between two hidden states form anything else then the observations. A hidden state will therefore be created for every sample in the HMM to keep the model in 1st order and keeping the Markov property.

In a hidden Markov model a node is connected with a direction to other nodes with a certain probability. This denotes the probability of transition from one state to another. There are two different forms of transition in such a model.
First there is the hidden state transition from one state to another here denoted as $a_{ij}$, meaning the transition from hidden state $x_i$ to hidden state $x_j$. It can also be described as, the probability that we know that the state is $x_i$ at time t, what is the probability that the next hidden state is $x_j$. We use the term hidden state transition for this.
Secondly we have the transition from a hidden state to a emission state, this is called emitting a observable state. It is noted as $b_{ij}$, meaning emitting emission state $e_j$ in hidden state $x_i$. This is also associated to a certain probability.
Furthermore we also have the initial distribution. This denotes the probability distribution of any hidden state without a prior hidden state, the initial probability for being in hidden state $x_i$. This can essentially be described as, the probability of starting in that state.
All of these types of transition and probability can be described in a matrix each, which brings us to the formal definition of a HMM:
\begin{align*}
\theta = \{A,B,\pi\} &|\\
A &= \text{hidden state transition matrix}\\
B &= \text{emission matrix}\\
\pi &= \text{initial distribution}
\end{align*}
Here we can see that we have a matrix for each form of probability. This means that each matrix is containing probabilities. The transition matrix A is the matrix denoting the hidden state transition probability $P(X_t=x_j|X_{t-1}=x_i)$. The emission matrix B is denoting the emission probability, $P(E_t=e_j|X_t=x_j)$. This is very similar to Bayesian networks and hidden Markov models are actually a specific form of Bayesian networks. We can therefore also graphically represent these two matrices in a causal network, this can be seen in \cref{fig:HMMCausalNet}.

\begin{figure}[htbp]
\centering
\begin{tikzpicture}[->]
  %Hidden Nodes
  \node[circle, draw=black, minimum size=1cm] (x1) {$x_1$};
  \node[circle, draw=black, minimum size=1cm, node distance=3cm, right=of x1] (x2) {$x_2$};
  \node[circle, draw=black, minimum size=1cm, node distance=3cm, below=of x1] (x3) {$x_3$};
  \node[circle, draw=black, minimum size=1cm, node distance=3cm, below=of x2] (x4) {$x_4$};

  %Emission states
  \node[circle] (e11) [draw=black, minimum size=1cm, node distance=1.5cm, above=of x1] {$e_1$};
  \node[circle] (e12) [draw=black, minimum size=1cm, node distance=1.5cm, left=of x1] {$e_2$};
  \node[circle] (e21) [draw=black, minimum size=1cm, node distance=1.5cm, above=of x2] {$e_1$};
  \node[circle] (e22) [draw=black, minimum size=1cm, node distance=1.5cm, right=of x2] {$e_2$};
  \node[circle] (e31) [draw=black, minimum size=1cm, node distance=1.5cm, left=of x3] {$e_1$};
  \node[circle] (e32) [draw=black, minimum size=1cm, node distance=1.5cm, below=of x3] {$e_2$};
  \node[circle] (e41) [draw=black, minimum size=1cm, node distance=1.5cm, below=of x4] {$e_1$};
  \node[circle] (e42) [draw=black, minimum size=1cm, node distance=1.5cm, right=of x4] {$e_2$};
  
  \path
  %Emission lines
    (x1) edge[out=90,in=-90] node[rectangle,fill=white] {$b_{11}$} (e11)
    (x1) edge[out=180,in=0] node[rectangle,fill=white] {$b_{12}$} (e12)
    (x2) edge[out=90,in=-90] node[rectangle,fill=white] {$b_{21}$} (e21)
    (x2) edge[out=0,in=180] node[rectangle,fill=white] {$b_{22}$} (e22)
    (x3) edge[out=180,in=0] node[rectangle,fill=white] {$b_{31}$} (e31)
    (x3) edge[out=-90,in=90] node[rectangle,fill=white] {$b_{32}$} (e32)
    (x4) edge[out=-90,in=90] node[rectangle,fill=white] {$b_{41}$} (e41)
    (x4) edge[out=0,in=180] node[rectangle,fill=white] {$b_{42}$} (e42)

  %Transition lines
    (x1) edge[out=15,in=165] node[rectangle,fill=white] {$a_{12}$} (x2)
    (x2) edge[out=-165,in=-15] node[rectangle,fill=white] {$a_{21}$} (x1)
    (x2) edge[out=-75,in=75] node[rectangle,fill=white] {$a_{24}$} (x4)
    (x4) edge[out=105,in=-105] node[rectangle,fill=white] {$a_{42}$} (x2)
    (x4) edge[out=-165,in=-15] node[rectangle,fill=white] {$a_{43}$} (x3)
    (x3) edge[out=15,in=165] node[rectangle,fill=white] {$a_{34}$} (x4)
    (x3) edge[out=105,in=-105] node[rectangle,fill=white] {$a_{31}$} (x1)
    (x1) edge[out=-75,in=75] node[rectangle,fill=white] {$a_{13}$} (x3)
  
    (x1) edge[out=-30,in=120] node[rectangle,fill=white] {$a_{14}$} (x4)
    (x4) edge[out=150,in=-60] node[rectangle,fill=white] {$a_{41}$} (x1)
    (x3) edge[out=60,in=-150] node[rectangle,fill=white] {$a_{32}$} (x2)
    (x2) edge[out=-120,in=30] node[rectangle,fill=white] {$a_{23}$} (x3)

    (x1) edge [out=150,in=120,loop,looseness=15] node[rectangle,fill=white] {$a_{11}$} (x1)
    (x2) edge [out=60,in=30,loop,looseness=15] node[rectangle,fill=white] {$a_{22}$} (x2)
    (x3) edge [out=-120,in=-150,loop,looseness=15] node[rectangle,fill=white] {$a_{33}$} (x3)
    (x4) edge [out=-30,in=-60,loop,looseness=15] node[rectangle,fill=white] {$a_{44}$} (x4);
\end{tikzpicture}
\caption[A hidden Markov model represented as a causal network.]{A hidden Markov model, with 4 hidden states and 2 emission states, represented as a causal network.}\label{fig:HMMCausalNet}
\end{figure}

\Cref{fig:HMMCausalNet} represents a HMM with 4 hidden notes and 2 emission states. The names on the arrows in the matrix (e.g. $a_{ij}$ and $b_{ij}$) represent entries in the two matrices, i.e. a probabilities. We can see that every hidden state is connected to every hidden state (including itself) and that every hidden state can emit every emission state. This is not always the case in really, since some hidden states could never directly reach other states and not every emission state could be emitted from every hidden state. This could be represented either as a sparse matrix or a matrix with entries with the value 0. In essence, they are equal, since a connection that has a probability of 0 can never happen, hence, there is no connection. This is just a case of how one represents \enquote{no connection}. Both of these representations pose a problem. The learning algorithms for HMMs cannot learn the connections between states, meaning that if there is no connection to begin with, there never will be in the future, no matter what. This is a dangerous assumption to make since we rarely have a complete understanding of the problem domain (after all, it is why we are using machine learning) and/or have a complete understanding of what each state exactly represent. In some cases a complete understanding of these things is not necessary, some things are simply just impossible. This cannot be said about this project; firstly all hidden states can emit all emission states. Since we have a world with unreliable sensors and what we think is an impossible world/transition at this time could simply be faulty sensors way of representing it. Further more, addressing the transition between hidden states, the hidden states are partly based on the users memory and how he/she is thinking. This can change from one state to another at will, even if the transition is physically impossible. As an initial assumption we have that the system thinks the user remembers the events in the past as they happened but since a user can forget, see reality differently or our sensors observe wrongly, we leave all connections in the model and thereby making it able to accommodate this change and unreliability.

In the formal definition of HMM we also had $\pi$, a matrix denoting the initial hidden state probability distribution, $P(X_1=x_i)$. Until now we have only illustrated models with relative time, but the initial probabilities gives us the notion that there is an absolute time starting at time 1 with a probability of this table. This is needed since any observation must have a starting point, and is going to become very important later when we will learn and reason about the model. We can now expand the trellis diagrams with this notion of a starting point as can be seen in \cref{4thMarkovModel}.

\begin{figure}[htbp]
\centering
\begin{tikzpicture}
  %Hidden Nodes
  \node[ellipse] (h1)   [draw=black, minimum width=1cm, minimum height=.75cm] {$x_1$};
  \node[ellipse] (h2)   [draw=black, minimum width=1cm, minimum height=.75cm, right=of h1] {$x_2$};
  \node          (dots) [draw=none,  minimum width=1cm, minimum height=.75cm, right=of h2] {\LARGE \dots};
  \node[ellipse] (htp1) [draw=black, minimum width=1cm, minimum height=.75cm, right=of dots] {$x_{t-1}$};
  \node[ellipse] (ht)   [draw=black, minimum width=1cm, minimum height=.75cm, right=of htp1] {$x_t$};
  \node[ellipse] (htf1) [draw=black, minimum width=1cm, minimum height=.75cm, right=of ht] {$x_{t+1}$};

  %Emission states
  \node[ellipse] (e1)   [draw=black, minimum width=1cm, minimum height=.75cm, below=of h1] {$e_1$};
  \node[ellipse] (e2)   [draw=black, minimum width=1cm, minimum height=.75cm, below=of h2] {$e_2$};
  \node[ellipse] (etp1) [draw=black, minimum width=1cm, minimum height=.75cm, below=of htp1] {$e_{t-1}$};
  \node[ellipse] (et)   [draw=black, minimum width=1cm, minimum height=.75cm, below=of ht] {$e_t$};
  \node[ellipse] (etf1) [draw=black, minimum width=1cm, minimum height=.75cm, below=of htf1] {$e_{t+1}$};
  
  %Emission lines
  \draw [->, to path={-| (\tikztotarget)}] (h1) edge[out=-90,in=90] (e1);
  \draw [->, to path={-| (\tikztotarget)}] (h2) edge[out=-90,in=90] (e2);
  \draw [->, to path={-| (\tikztotarget)}] (htp1) edge[out=-90,in=90] (etp1);
  \draw [->, to path={-| (\tikztotarget)}] (ht) edge[out=-90,in=90] (et);
  \draw [->, to path={-| (\tikztotarget)}] (htf1) edge[out=-90,in=90] (etf1);
  
  %1. order lines
  \draw [->, to path={-| (\tikztotarget)}] (h1) edge[out=0,in=180] (h2);
  \draw [->, to path={-| (\tikztotarget)}] (h2) edge[out=0,in=180] (dots);
  \draw [->, to path={-| (\tikztotarget)}] (dots) edge[out=0,in=180] (htp1);
  \draw [->, to path={-| (\tikztotarget)}] (htp1) edge[out=0,in=180] (ht);
  \draw [->, to path={-| (\tikztotarget)}] (ht) edge[out=0,in=180] (htf1);
\end{tikzpicture}
\caption[Absolute and relative time trellis diagram for hidden Markov model]{An absolute and relative time trellis diagram for a 1st order hidden Markov model}\label{fig:4thMarkovModel}
\end{figure}




The transition matrix is the matrix that.

We therefore need to en-cope the history in the hidden states of the model. 

the possibly to en-cope unobserved variables in the states via hidden states (explained later in this section)

An initial implementation of a Markov chain in our problem domain could make the states in the model snapshots of the sensors in the home. Here two problems arises. First, what a person decides to do next is not only dependent on what he or she is currently doing, also what has happen in the past.


A Markov chain can have multiple orders. This is how many . Another problem also arises when using Markov models. Of how high order should the model be? A first order 

 In a Markov chain.
Hidden states: The real pattern the user is currently doing.
Evidence states: Snapshots of the problem domain.

Hidden Markov chains are the simplest form of

\subsection{Learning}
For learning the Baum-Welch algorithm, a specific implementation of the forwards-backwards algorithm, was chosen.

\subsection{Prediction}
