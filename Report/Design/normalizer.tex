\section{Normaliser}
\label{sec:normaliser}
The purpose of the normaliser is to reduce the size of data such that large data can be handled. The Normalisation process consists of two parts. The first part is responsible for clustering the data using the method known as the elbow method. The second part is responsible for optimisation of these clusters. The optimisation part is to reduce number the number of clusters and thereby increase the accuracy of the clusters by eliminating hard breaks. The second part implies a method known as Jenks natural breaks optimisation.

\subsection{The Elbow Method}
\label{sub:elbow_method}
The elbow method uses the variance of the each cluster to determine, how many clusters are the optimal amount. The idea is simple and can be described as follows:
$$ Start with k = 1   Where k is numbers of clusters and k>0
$$ Increase k until the percentage difference in variance drops then
optimal number of k is reached$$
One start with a number clusters and then variance of each cluster is calculated and compared across the clusters. The variance is calculated in this fashion: 
$$ \sum_{i = 1}^{n}x_{i}-\overline{x}$$
\subsection{Jenks' Natural Breaks Optimisation}
\label{sub:jenks} 