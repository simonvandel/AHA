\subsubsection{Experiments}

\paragraph{Distance}

\subparagraph{Hypothesis}

According to the datasheet, the sensor can measure motion from 10 centimeters to
6 meters.

\subparagraph{Test procedure}

The sensitivity setting of the sensor essentially controls the distance. First,
the sensitivity is set to the lowest sensitivity. At that sensitivity, the
sensor should detect motion at a range of 10 centimeters, according to the above
hypothesis.

The sensor is measured by the test conductor approaching the sensor while
waving. Once the sensor detects the motion, the distance from the test conductor
to the sensor is measured. This is done three times to check for consistency. The sensitivity is then upped one unit on the sensor,
and the test conductor measures the distance for the new sensitivity.

\subparagraph{Results}

\subparagraph{Partial conclusion}



\paragraph{Delay time}

\subparagraph{Hypothesis}

According to the datasheet, the sensor has a delay time of 1 second to 25 seconds.

\subparagraph{Test procedure}

In this experiment, the sensor is set in the non-retriggerable setting (L
position), so the delay time is not extended on additional motion.

First the sensor is adjusted to have the maximum delay time of 25 seconds
according to the above hypothesis. This is then tested by triggering the sensor
and starting a stopwatch simultaneously. When the sensor stops signalling, the
stopwatch is stopped and the actual delay time noted. This is done three times
for each setting of delay time to check for consistency.

The next iteration of the experiment, the delay time potentiometer on the sensor
is rotated one unit, decreasing the value.

\subparagraph{Results}

\subparagraph{Partial conclusion}