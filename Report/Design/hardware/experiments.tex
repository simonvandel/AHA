\subsection{Experiments}
\label{sub:Experiments}
In order to determine the reliability of the hardware several experiments are conducted with the PIR sensor.
The tests are all based on data found in the datasheet for the hardware.
The main purpose of the tests is to determine the correlation between real world performance and the hardware specifications
found on the datasheet.
\\\\
Four tests are carried out with one main setup that are applied to all of them. The four consists of the following:

\begin{enumerate}
  \item Sensitivity Test
  \item Angle Test
  \item Distance Test
  \item Delay Time Test
\end{enumerate}
\subsubsection{The Setup}
\label{subs:The Setup}
As aforementioned the setup for all the experiments are identical
that is to ensure consistency of the results by reducing variables, that affects
the results, as much as possible.  The PIR sensor is placed in place surrounded wall
and where other moving objects are limited to a certain degree. In order to ensure that the
results are as precise as possible, all four tests are carried out at the same exact place but
the method of the individual test differs.
The PIR sensor is placed in a certain height and made sure the sensor itself is
sitting still during the tests.
\paragraph{Sensitivity}
\label{par:Sensitivity}

\subparagraph{Hypothesis}
\label{subp:SenHypothesis}
According to the datasheet the PIR sensor will be too sensitive if the distance potentiometer is rotated counter-clockwise.
The sensor should get so sensitivethat the PIR sensor would be triggered by the atmosphere even
if no moving object is existing\cite{datasheet_pir1}.
\subparagraph{Test Procedure}
\label{subp:SenTest Procedure}
The PIR sensor is configured with the distance potentiometer rotated
counter-clockwise as far as possible and the first triggered to test that everything works.
After the sensor is configured every object that can affect the sensor is removed and the sensor is observed.
The observation is to determine whether the sensor will detect objects are it is not
supposed to detect and how sensitive the sensor really can is.
\subparagraph{Results}
\label{subp:SenResults}

\subparagraph{Partial Conclusion}
\label{subp:SenPartial Conclusion}


\paragraph{Angle}
\label{par:Angle}

\subparagraph{Hypothesis}
\label{subp:AngHypothesis}
According to the datasheet of the PIR sensor,
the sensor has a detecting angle of 120 degrees.
The sensor should according to the datasheet not detect any angle above 120 degrees.

\subparagraph{Test Procedure}
\label{subp:AngTest Procedure}
The sensor is placed in a certain position and detecting angle is measured equivalent to
the specified angle found in the datasheet.
For the sake of explanation let the angle above 120 degree be the grey zone and
the detecting area that is within the 120 degree white zone.
A moving object is placed in the grey zone and slowing moving into the white zone.
The PIR sensor should be triggered when the moving object enters the white zone in order
to work exactly as specified in the hardware specifications.
The zone of the moving object and PIR sensor is observed accordingly
and thereby the actual detecting angle is determined.
\subparagraph{Results}
\label{subp:AngResults}

\subparagraph{Partial Conclusion}
\label{subp:AngPartial Conclusion}

\paragraph{Distance}

\subparagraph{Hypothesis}

According to the datasheet, the sensor can measure motion from 10 centimeters to
6 meters.

\subparagraph{Test procedure}

The sensitivity setting of the sensor essentially controls the distance. First,
the sensitivity is set to the lowest sensitivity. At that sensitivity, the
sensor should detect motion at a range of 10 centimeters, according to the above
hypothesis.

The sensor is measured by the test conductor approaching the sensor while
waving. Once the sensor detects the motion, the distance from the test conductor
to the sensor is measured. This is done three times to check for consistency. The sensitivity is then upped one unit on the sensor,
and the test conductor measures the distance for the new sensitivity.

\subparagraph{Results}

\subparagraph{Partial conclusion}


\paragraph{Delay time}

\subparagraph{Hypothesis}

According to the datasheet, the sensor has a delay time of 1 second to 25 seconds.

\subparagraph{Test procedure}

In this experiment, the sensor is set in the non-retriggerable setting (L
position), so the delay time is not extended on additional motion.

First the sensor is adjusted to have the maximum delay time of 25 seconds
according to the above hypothesis. This is then tested by triggering the sensor
and starting a stopwatch simultaneously. When the sensor stops signalling, the
stopwatch is stopped and the actual delay time noted. This is done three times
for each setting of delay time to check for consistency.

The next iteration of the experiment, the delay time potentiometer on the sensor
is rotated one unit, decreasing the value.

\subparagraph{Results}

\subparagraph{Partial conclusion}
