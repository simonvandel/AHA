\subsubsection{User Interfaces}
There are two kinds of user interaction in the system. We will here be calling them passive and active interaction.

\paragraph{Passive User Interaction}
Passive interaction is the interaction the user makes with the system that he/she should not notice as interaction. This is when he or she turns off the light on a switch on the system as the user normally would on the wallplug or turns on the television via the remote, and alike. Passive user interaction is very good since it very well holds against the system needing to be easy to use stated in \cref{}\jenote{Referer til requirements når vi får lavet dem} because the user only needs to keep doing what he or she has always been doing and will normally not even notice that he or she is interacting with the system but the system still get the needed information.

\paragraph{Active User Interaction}
The active user interaction is the interaction happening with the user actively trying to control or affect the system. The problem with this interaction is that it has the potential of going against the principle of easy user interaction and caution is needed here to make it as easy to use and needed as little as possible.

\paragrath{}

This interaction is done through a user interface on for instance the central unit in the system see \cref{System Interfaces} or a PC connected to the system.


The requirement for invisible for the user as possible. Meaning that the optimal user interface for the system would be no user interface, rendering  interaction from the user needless. Since the system is relying on machine learning to advance its knowledge about the use patterns of the user, and any system in a learning environment is able to make mistakes and must be able to learn from these, the user needs some way to inform the system that it has made a mistake. The interface could also be able to inform the system when it has made a correct action but since the system should be conservative in its behaviour meaning that it is far better to let the user keep doing the tasks than doing something that is against the users wishes. Therefore the system will be doing a lot more already known correct tasks and adjustments to them. Furthermore the goal of the system is to only right choices. Both of these principles would mean that if the user needed to reward correct behaviour the system would need constant or at least a lot more interaction, thereby going against the principle of invisibility. The system will therefore assume no interaction means correct behaviour.
 
To be able to make the interface as invisible as possible a goal for the interface is to try to incorporate it into the already.
