\subsection{Architecture}

\Cref{sec:architecture} describes the overall architecture of the system that solves the three tasks; data collection, analysis of data and mimicking patterns learned from the analysis. This section will zoom in on the architecture designed to perform the second and third task. This is the artificial intelligence (AI) subsystem of the system.

% DRAWING

As can be seen from the drawing \cref{}, the AI subsystem contains 6 modules: communication, normaliser, sampler, database, learner and reasoner. The rationale of this particular architecture and its modules is explained in the following.

\subsubsection{Communication}
The input to the AI subsystem is a stream of sensor value measurements. These packets of sensor values, from now on called snapshots, arrive continuously as devices from the sensor/actuator subsystem send them. The communication module handles these packets and also provides functionality to send data to devices in the sensor/actuator subsystem. When snapshots arrive at the communication module, the data of the packet is encoded in the binary format described in \cref{}\sinote{link til encoding beskrivelse}. The communication module decodes the binary data and notes the time the packet was received.

The decoded sensor values and the time of arrival is then send to the normaliser.

\subsubsection{Normaliser}
To remove noise from the sensors and to reduce the size of the problem, a normalisation pass is performed on the input. The normalisation pass is explained deeper in \cref{}\sinote{referer til normalisation design}. The output of the normalisation is the same set of sensor values, but normalised to fit into a domain of size k. The time of arrival is not normalised.

The normalised snapshots and the times of arrival are then fed to the sampler.

\subsubsection{Sampler}
This module performs three tasks:

\begin{enumerate}
\item It connects snapshots with the previous n snapshots. The previous n snapshots represent the history of the snapshot fed to the sampler. Together the current snapshot and its history of snapshots form a state scope.
\item It calculates whether an action has occured from the previous snapshot to the current. An action has occured when one or more emulatable sensors have changed its value.
\item It hashes the state scope.
\end{enumerate}

The hashed state scopes and the calculated action aswell as the time of arrival is now sent to 2 modules: the database and the reasoner.

\subsubsection{Database}
\sinote{TODO}