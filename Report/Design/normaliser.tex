\section{Normaliser}
\label{sec:normaliser}
In order to reduce the size of data and minimize noise the data is passed through a normaliser. The normaliser has the responsibility to shrink and standardise the data with a standardisation method that is used for normalisation of statistics.
\\\\
The method consists of a standardisation formula, that indexes the input data into a number between $0 and 1$.
The standardisation formula is given as 
$$Norm_{i} =\frac{X_i-X_{min}}{X_{Max}-X_{Min}}| 1 < Norm_i < 0 ,  X_i is data point i, X_{Min} is the minimum among the data, X_{Max} is maximum among the data and Norm_i is the normalised data point i$$

The formula calculates a decimal number between 0 and 1 the decimal number is rounded and indexed into a number between 0 and 9. The indices are determined by the first decimal of the $Norm_i$ e.g. $if Norm_i = 0.12$ then the index would be $1$. One can say that the decimal will be multiplied by 10 and rounded to the closest integer. 