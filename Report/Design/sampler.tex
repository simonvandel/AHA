%mainfile: ../master.tex
\section{Sampler}\label{sec:sampler}
The job of the sampler is to map the current sensor values, with a given history of snapshots, into some input features for the learner.
The idea is to find correlations between sensor values at a given time $t$, rather than finding correlations between snapshots from time $t_{n-k}$ to $t_n$. This is also the reason for making a new sample every time a new normalised snapshot is received, as first order Markov models inherently have the Markov property\footnote{The probability distribution of the next state depends only on the current state and not on the sequence of events that preceded it.\cite{wiki_markov_chain}}. Including the entire history will violate the Markov property as discussed in \cref{sec:hmm}. A solution to this problem is, as discussed in \cref{sec:hmm}, to encode the history in a state. This section further describes how the samples are made.

Some research on how many states a usage pattern stretches over, should be investigated, in order to find how many states, should be included in the scopes.

\subsection{Emulatable Actions}
One can infer that an action has happened in the problem domain, when an observed sensor value, which is also emulatable, changes from the state, given at time $t-1$ to the state given at time $t$. The sampler can then, along with the history of the states, that lead to this inferred action, pass the action itself, to the learner. The format of an action is a 3-tuple, $(S,V_1,V_2)$ describing sensor $S$ affected by values $V_1, V_2$(the sensors values at time $t-1$ and $t$).

\subsection{Unique Identifier for Snapshot}

In order to differentiate one snapshot from another, without having to observe each individual sensor value in the snapshots, a unique identifier for the snapshot is required.
This approach has some benefits, but also some limitations. The learning algorithm has less of an opportunity to over fit on a certain noisy sensor value in the problem domain, in that it can not directly observe the individual sensor values from each other.
Another benefit is that by concatenating all the sensor values into one feature, we reduce the number of features the learner has as input for the problem, on the other hand we increase the value domain of each of the features, to differentiate snapshot from each other.

\subsection{Loss of Resolution}
A disadvantage of using a unique identifier for snapshots is that it is impossible to reason about the actual sensor values in the snapshots, as those are removed when using a unique identifier. It is impossible to determine the \enquote{distance} between two snapshots, as there is no ordering of the unique identifiers used for snapshots.