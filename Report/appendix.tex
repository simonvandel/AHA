\appendix
\label{appendix_start}

\chapter{Listings}

\lstset{language=C}
\begin{lstlisting}[label = lst:arduinoPhotoCode, caption = Arduino program for photoresistor tests]
#define sensor1Pin 0
#define sensor2Pin 1
void setup() {
  Serial.begin(9600);
}

void loop() {
  int val1 = analogRead(sensor1Pin);
  int val2 = analogRead(sensor2Pin);
  Serial.print(val1);
  Serial.print(",");
  Serial.print(val2);
  Serial.print("\n");
}
\end{lstlisting}

\lstset{language=[Sharp]C}
\begin{lstlisting}[label = lst:cShPhotoCode, caption = C\# data processing code, showstringspaces=false]
using System;
using System.IO;

namespace ArduinoPhotoTest {
  class Program {
    [STAThreadAttribute]
    static void Main(string[] args) {
      StreamReader file = new StreamReader(@"Path/To/File.txt");
      string line;
      int a0Min = 99999;
      int a1Min = 99999;
      int a0Avg = 0;
      int a1Avg = 0;
      int a0Max = 0;
      int a1Max = 0;
      int counter = 0;
      while ((line = file.ReadLine()) != null) {
        int a0 = Convert.ToInt32(line.Split(',')[0]);
        int a1 = Convert.ToInt32(line.Split(',')[1]);
        a0Avg += a0;
        a1Avg += a1;
        if (a0Min > a0) a0Min = a0;
        else if (a0Max < a0) a0Max = a0;
        if (a1Min > a1) a1Min = a1;
        else if (a1Max < a1) a1Max = a1;
        counter++;
      }
      a0Avg /= counter;
      a1Avg /= counter;
      double a0Diff = ((double)a0Max / (double)a0Min) * 100d - 100;
      double a1Diff = ((double)a1Max / (double)a1Min) * 100d - 100;
      string text = string.Format("a0Avg = {0}\na0Min = {1}
                                  \na0Max = {2}\na0Diff = {6}
                                  %\na1Avg = {3}\na1Min = {4}
                                  \na1Max = {5}\na1Diff = {7}%",
        a0Avg, a0Min, a0Max, a1Avg, a1Min, a1Max, a0Diff, a1Diff);
      Console.WriteLine(text);
      System.Windows.Forms.Clipboard.SetText(text);
      Console.Read();
    }
  }
}
\end{lstlisting}

\chapter{CD}\label{app:cd}

% \framebox[125mm][cm]{}
\begin{figure}[htbp]
\centering
\begin{tikzpicture}
\centering
\node (rect) at (0,0) [loosely dotted, draw,thick,minimum width=125mm,minimum height=125mm] {};
\end{tikzpicture}
\end{figure}