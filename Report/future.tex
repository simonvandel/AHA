\chapter{Future Work}
\label{chap:future_work}
This chapter is about improvements to the project, which could be done in the future, as well as further discussion about the choices made in the project.

\section{Security}
The system is considered to be somewhat secure, since no information is transmitted across the internet, but only using XBee modules. However since the communication is through XBee modules, there is a certain risk that the packets sent can be sniffed with external XBee modules. If a person with a laptop and an XBee module is present near the system, it is possible for the person to sniff the packets. Since the packets are not encrypted, it can easily be decoded, thus the personal information can be exposed. An encryption of the packets would alleviate this security problem.

\section{Other Areas of the Home}
The system, as it is currently, only controls lights and appliances, due to a lack of time and resources. A huge improvement could be done in this area if the system would be able to control and monitor additional sensors and actuators, in order to affect the problem domain further.

\section{Improved Encoding Scheme}
The encoding scheme described in \cref{sec:encoding}, has no system of versioning of encoding schemes. Let's say a change is made to the encoding scheme. Now, some devices in the system might use the new scheme while others are not yet updated, so they use the old scheme. If only 1 version of the scheme can be in use in the system at a time, the system would break if not all devices used the same version of the system.

To allow for multiple versions of the scheme to be in use simultaneously, versioning could be added to the encoding scheme. This could be done by adding a \enquote{magic number} in the header. For example, \enquote{0xFA00} would mean version 0 of the scheme, \enquote{0xFA01} would mean version 1 of the scheme. Another benefit of adding this magic number is that when decoding the encoded data, the data can be somewhat validated by checking whether the data contains the defined magic number.

\section{Multiple Users}
The system is designed to learn the use patterns of the user. However due to lack of proper sensors, able to identify a specific user, the system cannot differentiate between users and thus the system can only function properly with a single user, as two users with conflicting patterns would confuse the system. One of the major future improvements would therefore be to make the system applicable for multiple users. This can be achieved if the system learns to differentiate between users. This process would need additional sensors. The learning algorithm would use the sensors, which differentiate users, to learn patterns for specific users.

\section{Tracking Ability}
The system could improve, if it had the ability to track the user and take actions, based on where the user is. The tracking could be through the user's smartphone or any other mobile device. The tracking ability would make the system able to take actions, knowing that the user is nearby. Those actions could be  convenient, such as turning on the coffee machine 5 minutes before the user arrives, meaning that the user does not have to wait for the coffee to brew.

\section{Weighted History}
To ensure that the system will be conservative, and not adapt to irregular changes immediately, some weighting of already known patterns should be done. This is to ensure that the learner is not confused by temporary irregularities, e.g. if a guest is present in the user's house, the usage pattern of the guest is not known to the system. As the guest is only temporarily present, the system should not adapt to this users patterns instead.

This conservativeness could be achieved by lowering the weight of newly learned usage patterns. In that way, temporary irregularities would not be suggested as actions immediately.
