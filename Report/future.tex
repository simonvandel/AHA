\chapter{Future Work}
\label{chap:future_work}
This chapter is about improvements that could be done in the future as well as further discussion about the choices made in the project. These improvements could be in the following areas.

\section{Security}
The system is considered secure, in the way that no information is transmitted across the internet, only locally using XBee modules. However since the communication is through XBee modules, there is a certain risk that the packets, sent and received internally, can be sniffed with external XBee modules. If a person with a laptop and an external XBee module is present near the system, it is possible for the person to sniff the internal packets. Since the packets are not encrypted it can easily be decoded thus the personal information can be exposed. An internal encryption of the packets, sent and received, would solve this security problem.

\section{Other Areas of the Home}
The system, as it is for now, only controls lights and appliances, due to the lack of time and resources. A huge improvement could be done in this area if the system would be able to control and monitor additional sensors and actuators, to monitor and affect the problem domain further.

\section{Improved encoding scheme}
Versions of protocol scheme

Checksum for invalidating packet

\section{Multiple Users}
The system is designed to learn the use patterns of the user. However due to lack of proper sensors to identify a specific user, the system cannot differentiate between users and thus the system can only be applied to one user at a time. This also means that two users with conflicting patterns will confused the system. One of the major future improvements would therefore be to make the system applicable for multiple users. This can be achieved if the system learns to differentiate between users by some sort of identity. The identity must be unique to the user. This process would need additional sensors. Additionally, the learning algorithm would use the sensors differentiating users to learn patterns specific for specific users.

\section{Tracking Ability}
The system could improve if it had the ability to track the user and take actions based on where the user is. The tracking could be through the user's smartphone or any other mobile device. The tracking ability would make the system able to take actions knowing that the user is nearby. Those actions could be trivial yet convenient, such as turning on the coffee machine 5 minutes before the user arrives, meaning the user does not have to wait.

\section{Weighted History}
To ensure that the system will be conservative, adapt to irregular changes immediately, some weighting of already known patterns should be done. This is to ensure that the learner is not confused by temporary irregularities, e.g. if a guest is present in the user's house, the usage pattern of the guest is not known to the system. As the guest is only temporarily present, the system should not adapt to this users patterns instead. 

This conservativeness can be achieved by lowering the weight of newly learned usage patterns. In that way, temporary irregularities will not be suggested as actions immediately.
