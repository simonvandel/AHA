%mainfile: ../master.tex
\section{Encoding Schemes}
\label{sec:encoding}
In order for the embedded subsystem to communicate with the learning subsystem, as shown in \cref{sec:ai}, the data transmitted back and forth, must be encoded in some way. It is shown in \cref{sec:xbee}, that the transmission speed reduces gradually as packet size increases. This means that we cannot send packages bigger than 72 bytes \cref{xbee_latency} using the ZigBee protocol. Because of the responsiveness requirement, described in \cref{sec:requirements}, the encoding of the data should be as compact as possible, to minimise the transmission speed and to allow packages to be sent from the Arduino Uno without splitting data into more packages.

One could reuse existing binary encoding instead of designing a new one from scratch. An example of an existing binary encoding standard would be Protocol Buffers\footnote{\url{https://developers.google.com/protocol-buffers/}}. Using an existing encoding format, would provide \enquote{free} encoding and decoding methods, as only the specification of the data would need to be described. An advantage of designing an encoding from scratch is that there is more control over the size of the final encoding size. It also allows for domain specific knowledge to be exploited in the encoding. For example, if it is known that a sensor only has a binary value domain, 1 bit is enough to encode this.

The primary concern of choosing an encoding for the communication subsystem, is responsiveness of the system.
The responsiveness of the system is directly dependent on size of the packages, thus the encoding has a direct influence on the responsiveness. Since changes to what needs to be sent in the future can be made, the system needs to be maintainable. This can be done by applying a general encoding scheme. The problem with a general encoding scheme is that it cannot utilie specific properties of current sensors. Furthermore it was found, through testing, that certain bytes are not transferred. A specific encoding scheme can avoid these complications in a way that a general implementation cannot. Therefore a specific encoding scheme was chosen. An implementation of a simple domain specific encoding scheme is described below. This will cover schemes to and from the embedded subsystem and the learning subsystem. Note that the encodings assume little endianness. The system satisfies maintainability by making the encoding scheme a module. This can therefore easily be switched with another encoding module if necessary.

\subsection{From Embedded Subsystem To Learning Subsystem}
We assume that the embedded subsystem only needs to transmit sensor values of the following types:

\begin{itemize}
\item Binary values (0, 1) - For example a PIR sensor - 1 bit required
\item Values ranging from 0 to 1023 - For example a photoresistor - 10 bits required, as this is the precision for Arduino Uno analogue input pins
\item Values ranging from 0 to 4294967295 - For example a distance sensor, which stores the distance in a 32 bit integer - 32 bits required
\item A flag that describes whether the sensor is emulatable by the system
\end{itemize}
By this assumption we can make some domain specific implementation for the encoding.

This is the actual data that needs to be transmitted, so we call this the body of the packet. If just this body was sent as the packet, every value would have to be 32 bits values as the size of the values is not specified in the packet. To solve this problem, a header is prepended to the packet that contains some metadata, allowing for a more compact data representation.

The header contains the following fields:

\begin{itemize}
\item Number of non-binary values - 5 bits required. The Arduino Uno has 20 input pins, so a maximum of 20 sensors can be connected
\item Index (starting from 1) of the first emulatable non-binary sensor - 5 bits
\item This field is repeating. There are as many fields of this type as there are number of non-binary values. This field type describes the size of the values - 1 bit required, with the following meaning of the bits: 0: 10 bits, 1: 32 bits
\item Number of binary values - 5 bits required. The Arduino Uno has 20 input pins, so a maximum of 20 sensors can be connected
\item Index (starting from 1) of the first binary emulatable sensor value - 5 bits required
\end{itemize}

The worst case packet size for this encoding scheme is sending 6 non-binary values that each occupy 32 bits, and 14 binary values that each occupy 1 bit. This would require the header to be of size $5 + 5 + 6 * 1 + 5 + 5 = 26$ bits. The body would require size $6 * 32 + 14 * 1 = 206$. In total the size of the encoding would be $26 + 206 = 232$ bits $= 29$ bytes. This is a theoretical maximum size that would be reached only if we assume sensors only need 1 input pin on the Arduino. In any case the size fits into a ZigBee packet.

If the same values were to be sent naively, each value would occupy 32 bits, so the packet would be of size $20 * 32 = 640$ bits $=80$ bytes. This would not fit into a ZigBee packet.

The data can be decoded in the following way.

\begin{enumerate}
\item Read the bits representing the number of non-binary values in the header
\item Read the bits representing the index of the first emulatable non-binary sensor
\item For all number of non-binary values, read the field describing the size of the non-binary value
\item Read the bits representing the number of binary values in the header
\item Read the bits representing the index of the first emulatable binary sensor
\item Read the sensor values based on the sizes read in the header
\end{enumerate}

\subsection{From  Learning Subsystem To Embedded Subsystem}

The learning subsystem generates actions that the embedded subsystem has to emulate. An action is encoded in the following way.

The first 16 bits describe the actuator index on the embedded station we want to alter. The next 16 bits describe the desired value the actuator should be changed to.

The size of these packets is always 4 bytes. If multiple actions should be emulated, multiple action packets are sent.
