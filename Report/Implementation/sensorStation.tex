%mainfile: ../master.tex
\section{Sensor Station}\label{sub:sensorStation}

The sensor stations are responsible for collecting data from sensors placed around a vicinity and acting on lighting appliances around the vicinity. The sensor station should transmit the most recent collected data to the AI subsystem all the time. This should also be done in a timely fashion, according to the responsiveness requirement of the system, that states that a response should be reacted upon within 100ms. Hereafter we will go through a little worst case analysis of the sensor station.

Having more than one sensor station is supported by the ZigBee\cref{xbee_latency} protocol on the Xbee devices. It is important to note when using a CSMA-CA protocol, that multiple transmitting devices can potentially result in an additional small delay to the overhead of transmitting. The delay cannot exceed $\text{BACKOFF}$ though otherwise the packet is dropped. This additionally applies both ways from sensor station to the AI subsystem and back.

The calculations made in \cref{sec:xbee} shows that the theoretical worst case time to send a packet is $50.076$ ms.

Besides the communications overhead, the sensor station should also read from and act on sensor connected to it, \chnote{tilføj timings for at læse fra sensorer} as found \cref{} and \cref{}.

The loop of the sensor station:
\begin{enumerate}
  \item Read sensors
  \item Send sensor values
  \item Receive actions
  \item Act
\end{enumerate}

As far as scheduling, the implementation is a simple round robin executing each task in a predefined order. The execution time is measured and following results are found
 \begin{itemize}
  \item Read Sensor : DUMMY ms
  \item Send Sensor Values : DUMMY ms
  \item Receive Actions : DUMMY ms
  \item Act : DUMMY ms 
\end{itemize}

Which leads to a worst case execution time of the sensor station from read to act in the loop is ....\chnote{calculate} \chnote{this is a part wise conclusion, same should be done for the AI subsystem(through normaliser, sampler and reasoner) og så adderes sammen i rapportens conclusion}
