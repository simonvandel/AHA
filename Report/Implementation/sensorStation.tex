%mainfile: ../master.tex
\section{Embedded Subsystem}\label{sub:sensorStation}

The embedded subsystem is responsible for collecting data from sensors, placed around a vicinity, and acting on lighting appliances around the vicinity. The embedded subsystem should transmit the most recent collected data to the learning subsystem continuously. This should also be done in a timely fashion, according to the responsiveness requirement of the system, that states that a response should be reacted upon within 100 ms, see \cref{sub:people}. We will now look at a worst case analysis of the embedded subsystem.

Having more than one device in the embedded subsystem is supported by the ZigBee\cite{xbee_latency} protocol on the XBee devices. It is important to note when using a CSMA-CA protocol, that multiple transmitting devices can potentially result in an additional small delay to the overhead of transmitting. The delay cannot exceed $\text{BACKOFF}$, as otherwise the packet is dropped. This additionally applies both from the embedded subsystem to the learning subsystem and back.

The calculations made in \cref{sec:xbee} shows that the theoretical worst case time to send a packet is $50.076$ ms.

The embedded subsystem does the following tasks in a loop:
\begin{enumerate}
  \item Read sensors
  \item Encode sensor values
  \item Send sensor values
  \item Decode actions
  \item Perform actions
\end{enumerate}

As far as scheduling, the implementation is a simple round robin, executing each task in a predefined order. The execution time of each step in this loop is described below. This assumes a setup where the following sensors are connected: A PIR sensor, an ultrasonic sensor and a photoresistor.

 \begin{enumerate}
  \item \textbf{Read sensors} According to our measurements, an analogue read takes $0.116 ms$. A digital read/write takes $0.008 ms$. 

The code to read the PIR sensor makes $10$ calls to digitalRead. That is, it takes approx $0.08 ms$ to read the PIR sensor. 

The code to read the photoresistor makes $10$ calls to analogRead. This sums to approximately $1.16 ms$ for a reading of the photoresistor. 

The ultrasonic sensor makes $3$ calls to digitalWrite, and waits in total $12$ microseconds. In addition to that, the sensor has to wait for the echo to come back. Assuming that the ultrasonic sensor is $0.2$ m away from an object. The speed of sound is approximately $342 m/s$. So it will take $0.2 m / 342 m/s = 0.000584 s = 0.584 ms$ for the sound travelling one way. So $0.584 ms * 2 = 1.168 ms$ for a round trip. An ultrasonic reading in total takes $3 * 0.008 ms + 0.012 ms + 1.168 ms = 1.204 ms$. 

In total, it takes $0.08 ms + 1.16 ms + 1.204 ms = 2.444 ms$ to read all sensors
  \item \textbf{Encode sensor values} According to our measurements, an encoding of the above sensor values, takes $0.134 ms$. The sensor values are encoded into 8 bytes
  \item \textbf{Send sensor values} According to the theoretical calculations presented in \cref{sec:xbee}, it takes $T_{\text{air}}(8) + CCA = 0.8_{\text{header}} + 0.032*8 ms + 0.128 ms = 1.184 ms$ to send 8 bytes.
  \item \textbf{Decode actions} This is a very simple operation, so it is estimated to take less than $0.1 ms$
  \item \textbf{Perform actions} For every action, either an digital write or analogue write is performed. Assuming just 1 action is to be performed, and a digitalWrite is to be called, it will take $0.008 ms$.
\end{enumerate}

These timings lead to the following execution time of the embedded subsystem: $\text{read sensors} + \text{encode} + \text{send} + \text{decode} + \text{actuate} = 2.444 ms + 0.134 ms + 1.184 ms + 0.1 ms + 0.008 ms = 3.87 ms$. This execution time is not including the time it takes for the learning subsystem to calculate an action based on sensor data. Also, a theoretical calculation is used for calculating \enquote(send sensor values). It may actually take longer.
