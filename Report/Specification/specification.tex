%mainfile: ../master.tex
\chapter{Specification}

<<<<<<< HEAD
<<<<<<< HEAD
This chapter is documentation of the requirement engineering process, as describe in \cite{sommerville}. The structure will follow the elicitation process of requirements, by the methods included in sommerville's book. Some contributing methods used to describe the problem domain are borrowed from \cref{OOAD}. 
=======
This chapter is documentation of the requirement engineering process, as describe in \citep{sommerville}. The structure will follow the elicitation process of requirements, by the methods included in sommerville's book. Some contributing methods used to describe the problem domain are borrowed from \cref{OOAD}. 
=======
This chapter is documentation of the requirement engineering process, as describe in \cite{sommerville}. The structure will follow the elicitation process of requirements, by the methods included in sommerville's book. Some contributing methods used to describe the problem domain are borrowed from \cref{OOAD}. 
>>>>>>> More user scenarios

<<<<<<< HEAD
Brainstorm of ideas/requirements:
<<<<<<< HEAD
>>>>>>> Specification refactor

%mainfile: ../master.tex
\section{Classes}
From the system definition \cref{sec:systemDefinition} we can extract some classes. A class is a \enquote{A description of a collection of objects with the same structure, behaviour and attributes} as described in \cite{OOAD}. Candidates for classes are found by observing nouns, adjectives and verbs present in the problem domain.

These classes will be beneficial when figuring out what is relevant in our model of the problem domain. This system should be able to keep an always up to date model of the problem domain. The list below summarises some candidates for classes that the system uses to structure this model.

\begin{itemize}
\item Person - The user of the system. The problem domain can contain multiple persons with different usage patterns, so the system should be able to differentiate them
\item Action - An action is a description of a specific behavior in the problem domain
\item Pattern - A sequence of dependent repetetive actions in the problem domain
\item Decision - A sensible choice made by the system based on knowledge gathered from the problem domain
\item Feedback - An evaluation of a decision the system has made
\item Sensor - The observational part of the system. Can observe actions that happen in the problem domain
\item Actuator - The acting part of the system. Can influence the problem domain
\item Location - A specific place in the problem domain
\item Time - At what time, a given stimuli has happened
\end{itemize}


%mainfile: ../../master.tex
\section{Events}

\begin{table}[h!]
  \centering
  \begin{adjustbox}{max width=\textwidth}
    \begin{tabular}{*{15}{|l}|}
        \hline
        \textbf{Event} & Person & Action & Pattern & Mistake & Sensor & Switch & Lamp & Light intensity & Room & Premise & Time \\
        \hline
        Entered home & \cmark & \cmark & \cmark & & \cmark & & \cmark & \cmark & \cmark & \cmark & \\
        \hline
        Left home & \cmark & \cmark & \cmark & & \cmark & & \cmark & & \cmark & \cmark & \\
        \hline
        Enter room & \cmark & \cmark & \cmark & & \cmark & & \cmark & \cmark & \cmark & & \\
        \hline
        Left room & \cmark & \cmark & \cmark & & \cmark & & \cmark & \cmark & \cmark & &\\
        \hline
        Flicked switch & \cmark & \cmark & \cmark & \cmark & \cmark & \cmark & \cmark & \cmark & \cmark & & \\
        \hline
        Outside/External light intensity down & \cmark & & & & \cmark & \cmark & \cmark & \cmark & \cmark & &\\
        \hline
        Outside/External light intensity up & \cmark & & & & \cmark & \cmark & \cmark & \cmark& \cmark & &\\
        \hline
        Person sleeping & \cmark & \cmark & \cmark & & \cmark & & \cmark & & \cmark & & \cmark\\
        \hline
    \end{tabular}
  \end{adjustbox}
  \caption{Event table}
  \label{tab:eventtable}
\end{table}

\begin{figure}
   \centering
   \begin{adjustbox}{max width=\textwidth}
    \includegraphics{Behaviour.pdf}
   \end{adjustbox}
   \caption{Behaviour diagram}
\end{figure}


%mainfile: ../master.tex
\section{User scenarios}\label{sec:userscenarious}


This section will describe branches of possible narratives derived from the use cases, described in the \cref{sec:usecases}.

<<<<<<< HEAD
<<<<<<< HEAD
<<<<<<< HEAD
=======
>>>>>>> 9c8f5030b9365c258596d255a2898784b52995cb
\textit{A user is reading a book, the light intensity of the sun is low now due the time of the day, and the user is having trouble reading. The system has learned that at this light intensity when the user is situated at the couch with tv not running, the user now wants the light to be on.}

\textit{A user is home alone in his/her living room. Now leaving a room to go the toilet but does not turn of the light in the living room, while on the toilet the user closes the door and does not observe the light in the living room being turned off by the system to conserve/save energy.}

\textit{A user just left home for a vacation, the user is leaves a few lights on due to stress, the system recognises this and switches the lights off. The user has not been home for 24 hours, the system now initiates in doing short cycles of simulating normal behaviour of the user in some rooms to seed the impression of someone being home, to proactively prevent attracting interest from any observing burglars.}

\textit{In a financial report an organisation recognises a substantial amount of funds is used on lighting the facility. The system generates a report on the lights around the facility, effective lighting hours. Which the supervisors can conclude on to argue were to invest on more energy efficient lighting. Further the system reports on broken or not functioning lighting.}


%Just for fun ? Many users is situated in the living room and the volume(db) of the stereo is really high, so users is shouting to hear each other and this occurs over prolonged duration, the system intellingently lowers the volume without the user noticing. And at 1am/pm(night) the user normally goes to sleep but not today, but because of neighbors the system lowers the volume further.
<<<<<<< HEAD
=======
A user is reading a book, the light intensity of the sun is low now due the time of the day, and the user is having trouble reading. The system has learned that at this light intensity when the user is situated at the couch with tv not running, the user now wants the light to be on.
=======
\textit{A user is reading a book, the light intensity of the sun is low now due the time of the day, and the user is having trouble reading. The system has learned that at this light intensity when the user is situated at the couch with tv not running, the user now wants the light to be on.}
>>>>>>> More user scenarios

\textit{A user is home alone in his/her living room. Now leaving a room to go the toilet but does not turn of the light in the living room, while on the toilet the user closes the door and does not observe the light in the living room being turned off by the system to conserve/save energy.}

<<<<<<< HEAD
Just for fun ? Many users is situated in the living room and the volume(db) of the stereo is really high, so users is shouting to hear each other and this occurs over prolonged duration, the system intellingently lowers the volume without the user noticing. And at 1am/pm(night) the user normally goes to sleep but not today, but because of neighbors the system lowers the volume further.
>>>>>>> Specification refactor
=======
\textit{A user just left home for a vacation, the user is leaves a few lights on due to stress, the system recognises this and switches the lights off. The user has not been home for 24 hours, the system now initiates in doing short cycles of simulating normal behaviour of the user in some rooms to seed the impression of someone being home, to proactively prevent attracting interest from any observing burglars.}

\textit{In a financial report an organisation recognises a substantial amount of funds is used on lighting the facility. The system generates a report on the lights around the facility, effective lighting hours. Which the supervisors can conclude on to argue were to invest on more energy efficient lighting. Further the system reports on broken or not functioning lighting.}


%Just for fun ? Many users is situated in the living room and the volume(db) of the stereo is really high, so users is shouting to hear each other and this occurs over prolonged duration, the system intellingently lowers the volume without the user noticing. And at 1am/pm(night) the user normally goes to sleep but not today, but because of neighbors the system lowers the volume further.
>>>>>>> More user scenarios
=======
>>>>>>> 9c8f5030b9365c258596d255a2898784b52995cb



<<<<<<< HEAD
<<<<<<< HEAD
%mainfile: ../master.tex
\section{Use cases}\label{sec:usecases}

This section will describe ways of interacting with the prototypical system derived from the requirements.




%mainfile: ../../master.tex
\subsection{Structure}\label{sub:structure}

\cref{fig:structure} describe what classes are associated to what classes. This will be useful for the designers and system when designing and reasoning about the knowledge base \cref{sub:KB}.

\begin{figure}
  \label{fig:structure}
  \centering
  \begin{adjustbox}{max width=\textwidth}
    \includegraphics{Structure.pdf}
  \end{adjustbox}
  \caption{Structure of the problem domain}
\end{figure}



Elicited requirements: %Mangler argumentation
=======
\begin{itemize}
\item The light in a users home will be turned off when not in need for it
\item The light management invisible to the user, light is turned on before a user can observe this behaviour.
\item Will learn usage patterns of the user, turn on light at specific time of day or at a specific light intensity.
\end{itemize}

From the system definition and "rich picture" of the problem domain that the system will be operating, we can extract some classes, \quote{A description of a collection of objects with the same structure, behaviour and attributes}, as described in \cref{OOAD}. This classes will be beneficial when figuring out what we need to keep track of in the problem domain, or what is relevant in our model of the problem domain.

Classes:
\begin{itemize}
\item Person - The user of the system
\item Action - An action the system should recognise as feedback
\item Pattern - The usage patter of a user, what is typical behaviour
\item Mistake - System did something wrong by the user
\item Sensor - The observational member of the system
\item Switch - A feedback source
\item Lamp - The light source on which system action
\item Light intensity - Luminance
\item Room - What room is the user in
\item Premise - The user home enviroment
\item Time - At what time, a pattern property
\end{itemize}

\begin{table}[h!]
\centering
\begin{adjustbox}{max width=\textwidth}
\begin{tabular}{*{15}{|l}|}
    \hline
    \textbf{Event} & Person & Action & Pattern & Mistake & Sensor & Switch & Lamp & Light intensity & Room & Premise & Time \\
    \hline
    Entered home & \cmark & \cmark & \cmark & & \cmark & & \cmark & \cmark & \cmark & \cmark & \\
    \hline
    Left home & \cmark & \cmark & \cmark & & \cmark & & \cmark & & \cmark & \cmark & \\
    \hline
    Enter room & \cmark & \cmark & \cmark & & \cmark & & \cmark & \cmark & \cmark & & \\
    \hline
    Left room & \cmark & \cmark & \cmark & & \cmark & & \cmark & \cmark & \cmark & &\\
    \hline
    Flicked switch & \cmark & \cmark & \cmark & \cmark & \cmark & \cmark & \cmark & \cmark & \cmark & & \\
    \hline
    Outside/External light intensity down & \cmark & & & & \cmark & \cmark & \cmark & \cmark & \cmark & &\\
    \hline
    Outside/External light intensity up & \cmark & & & & \cmark & \cmark & \cmark & \cmark& \cmark & &\\
    \hline
    Person sleeping & \cmark & \cmark & \cmark & & \cmark & & \cmark & & \cmark & & \cmark\\
    \hline
\end{tabular}
\end{adjustbox}
  \caption{Test Table}
  \label{tab:label_test}
\end{table}
>>>>>>> merge
=======
%mainfile: ../master.tex
\section{Classes}
From the system definition \cref{sec:systemDefinition} we can extract some classes. A class is a \enquote{A description of a collection of objects with the same structure, behaviour and attributes} as described in \cite{OOAD}. Candidates for classes are found by observing nouns, adjectives and verbs present in the problem domain.

These classes will be beneficial when figuring out what is relevant in our model of the problem domain. This system should be able to keep an always up to date model of the problem domain. The list below summarises some candidates for classes that the system uses to structure this model.

\begin{itemize}
\item Person - The user of the system. The problem domain can contain multiple persons with different usage patterns, so the system should be able to differentiate them
\item Action - An action is a description of a specific behavior in the problem domain
\item Pattern - A sequence of dependent repetetive actions in the problem domain
\item Decision - A sensible choice made by the system based on knowledge gathered from the problem domain
\item Feedback - An evaluation of a decision the system has made
\item Sensor - The observational part of the system. Can observe actions that happen in the problem domain
\item Actuator - The acting part of the system. Can influence the problem domain
\item Location - A specific place in the problem domain
\item Time - At what time, a given stimuli has happened
\end{itemize}


%mainfile: ../../master.tex
\section{Events}

\begin{table}[h!]
  \centering
  \begin{adjustbox}{max width=\textwidth}
    \begin{tabular}{*{15}{|l}|}
        \hline
        \textbf{Event} & Person & Action & Pattern & Mistake & Sensor & Switch & Lamp & Light intensity & Room & Premise & Time \\
        \hline
        Entered home & \cmark & \cmark & \cmark & & \cmark & & \cmark & \cmark & \cmark & \cmark & \\
        \hline
        Left home & \cmark & \cmark & \cmark & & \cmark & & \cmark & & \cmark & \cmark & \\
        \hline
        Enter room & \cmark & \cmark & \cmark & & \cmark & & \cmark & \cmark & \cmark & & \\
        \hline
        Left room & \cmark & \cmark & \cmark & & \cmark & & \cmark & \cmark & \cmark & &\\
        \hline
        Flicked switch & \cmark & \cmark & \cmark & \cmark & \cmark & \cmark & \cmark & \cmark & \cmark & & \\
        \hline
        Outside/External light intensity down & \cmark & & & & \cmark & \cmark & \cmark & \cmark & \cmark & &\\
        \hline
        Outside/External light intensity up & \cmark & & & & \cmark & \cmark & \cmark & \cmark& \cmark & &\\
        \hline
        Person sleeping & \cmark & \cmark & \cmark & & \cmark & & \cmark & & \cmark & & \cmark\\
        \hline
    \end{tabular}
  \end{adjustbox}
  \caption{Event table}
  \label{tab:eventtable}
\end{table}

\begin{figure}
   \centering
   \begin{adjustbox}{max width=\textwidth}
    \includegraphics{Behaviour.pdf}
   \end{adjustbox}
   \caption{Behaviour diagram}
\end{figure}

>>>>>>> structure and usecases, splitted files

\begin{itemize}
\item The light in a users home will be turned off only when there is no need for it.
\item The user should be in doubt that the system will eventually turn off the light.
\item The should not be afraid that the system will suddenly turn off the light, whilst the user is need of it.
\item The light management should sought to be invisible to the user, light is turned on before a user can observe this behaviour.
\item Will learn usage patterns of the user, turn on light at specific time of day or at a specific light intensity.
\end{itemize}
=======
%mainfile: ../master.tex
\section{User scenarios}\label{sec:userscenarious}


This section will describe branches of possible narratives derived from the use cases, described in the \cref{sec:usecases}.

<<<<<<< HEAD
<<<<<<< HEAD
<<<<<<< HEAD
=======
>>>>>>> 9c8f5030b9365c258596d255a2898784b52995cb
\textit{A user is reading a book, the light intensity of the sun is low now due the time of the day, and the user is having trouble reading. The system has learned that at this light intensity when the user is situated at the couch with tv not running, the user now wants the light to be on.}

\textit{A user is home alone in his/her living room. Now leaving a room to go the toilet but does not turn of the light in the living room, while on the toilet the user closes the door and does not observe the light in the living room being turned off by the system to conserve/save energy.}

\textit{A user just left home for a vacation, the user is leaves a few lights on due to stress, the system recognises this and switches the lights off. The user has not been home for 24 hours, the system now initiates in doing short cycles of simulating normal behaviour of the user in some rooms to seed the impression of someone being home, to proactively prevent attracting interest from any observing burglars.}

\textit{In a financial report an organisation recognises a substantial amount of funds is used on lighting the facility. The system generates a report on the lights around the facility, effective lighting hours. Which the supervisors can conclude on to argue were to invest on more energy efficient lighting. Further the system reports on broken or not functioning lighting.}


%Just for fun ? Many users is situated in the living room and the volume(db) of the stereo is really high, so users is shouting to hear each other and this occurs over prolonged duration, the system intellingently lowers the volume without the user noticing. And at 1am/pm(night) the user normally goes to sleep but not today, but because of neighbors the system lowers the volume further.
<<<<<<< HEAD
=======
A user is reading a book, the light intensity of the sun is low now due the time of the day, and the user is having trouble reading. The system has learned that at this light intensity when the user is situated at the couch with tv not running, the user now wants the light to be on.
=======
\textit{A user is reading a book, the light intensity of the sun is low now due the time of the day, and the user is having trouble reading. The system has learned that at this light intensity when the user is situated at the couch with tv not running, the user now wants the light to be on.}
>>>>>>> More user scenarios

\textit{A user is home alone in his/her living room. Now leaving a room to go the toilet but does not turn of the light in the living room, while on the toilet the user closes the door and does not observe the light in the living room being turned off by the system to conserve/save energy.}

<<<<<<< HEAD
Just for fun ? Many users is situated in the living room and the volume(db) of the stereo is really high, so users is shouting to hear each other and this occurs over prolonged duration, the system intellingently lowers the volume without the user noticing. And at 1am/pm(night) the user normally goes to sleep but not today, but because of neighbors the system lowers the volume further.
>>>>>>> Specification refactor
=======
\textit{A user just left home for a vacation, the user is leaves a few lights on due to stress, the system recognises this and switches the lights off. The user has not been home for 24 hours, the system now initiates in doing short cycles of simulating normal behaviour of the user in some rooms to seed the impression of someone being home, to proactively prevent attracting interest from any observing burglars.}

\textit{In a financial report an organisation recognises a substantial amount of funds is used on lighting the facility. The system generates a report on the lights around the facility, effective lighting hours. Which the supervisors can conclude on to argue were to invest on more energy efficient lighting. Further the system reports on broken or not functioning lighting.}


%Just for fun ? Many users is situated in the living room and the volume(db) of the stereo is really high, so users is shouting to hear each other and this occurs over prolonged duration, the system intellingently lowers the volume without the user noticing. And at 1am/pm(night) the user normally goes to sleep but not today, but because of neighbors the system lowers the volume further.
>>>>>>> More user scenarios
=======
>>>>>>> 9c8f5030b9365c258596d255a2898784b52995cb


=======
%mainfile: ../master.tex
\section{User scenarios}\label{sec:userscenarious}


This section will describe branches of possible narratives derived from the use cases, described in the \cref{sec:usecases}.

<<<<<<< HEAD
<<<<<<< HEAD
<<<<<<< HEAD
=======
>>>>>>> 9c8f5030b9365c258596d255a2898784b52995cb
\textit{A user is reading a book, the light intensity of the sun is low now due the time of the day, and the user is having trouble reading. The system has learned that at this light intensity when the user is situated at the couch with tv not running, the user now wants the light to be on.}

\textit{A user is home alone in his/her living room. Now leaving a room to go the toilet but does not turn of the light in the living room, while on the toilet the user closes the door and does not observe the light in the living room being turned off by the system to conserve/save energy.}

\textit{A user just left home for a vacation, the user is leaves a few lights on due to stress, the system recognises this and switches the lights off. The user has not been home for 24 hours, the system now initiates in doing short cycles of simulating normal behaviour of the user in some rooms to seed the impression of someone being home, to proactively prevent attracting interest from any observing burglars.}

\textit{In a financial report an organisation recognises a substantial amount of funds is used on lighting the facility. The system generates a report on the lights around the facility, effective lighting hours. Which the supervisors can conclude on to argue were to invest on more energy efficient lighting. Further the system reports on broken or not functioning lighting.}


%Just for fun ? Many users is situated in the living room and the volume(db) of the stereo is really high, so users is shouting to hear each other and this occurs over prolonged duration, the system intellingently lowers the volume without the user noticing. And at 1am/pm(night) the user normally goes to sleep but not today, but because of neighbors the system lowers the volume further.
<<<<<<< HEAD
=======
A user is reading a book, the light intensity of the sun is low now due the time of the day, and the user is having trouble reading. The system has learned that at this light intensity when the user is situated at the couch with tv not running, the user now wants the light to be on.
=======
\textit{A user is reading a book, the light intensity of the sun is low now due the time of the day, and the user is having trouble reading. The system has learned that at this light intensity when the user is situated at the couch with tv not running, the user now wants the light to be on.}
>>>>>>> More user scenarios

\textit{A user is home alone in his/her living room. Now leaving a room to go the toilet but does not turn of the light in the living room, while on the toilet the user closes the door and does not observe the light in the living room being turned off by the system to conserve/save energy.}

<<<<<<< HEAD
Just for fun ? Many users is situated in the living room and the volume(db) of the stereo is really high, so users is shouting to hear each other and this occurs over prolonged duration, the system intellingently lowers the volume without the user noticing. And at 1am/pm(night) the user normally goes to sleep but not today, but because of neighbors the system lowers the volume further.
>>>>>>> Specification refactor
=======
\textit{A user just left home for a vacation, the user is leaves a few lights on due to stress, the system recognises this and switches the lights off. The user has not been home for 24 hours, the system now initiates in doing short cycles of simulating normal behaviour of the user in some rooms to seed the impression of someone being home, to proactively prevent attracting interest from any observing burglars.}

\textit{In a financial report an organisation recognises a substantial amount of funds is used on lighting the facility. The system generates a report on the lights around the facility, effective lighting hours. Which the supervisors can conclude on to argue were to invest on more energy efficient lighting. Further the system reports on broken or not functioning lighting.}


%Just for fun ? Many users is situated in the living room and the volume(db) of the stereo is really high, so users is shouting to hear each other and this occurs over prolonged duration, the system intellingently lowers the volume without the user noticing. And at 1am/pm(night) the user normally goes to sleep but not today, but because of neighbors the system lowers the volume further.
>>>>>>> More user scenarios
=======
>>>>>>> 9c8f5030b9365c258596d255a2898784b52995cb


>>>>>>> More user scenarios

%mainfile: ../master.tex
\section{Use cases}\label{sec:usecases}

This section will describe ways of interacting with the prototypical system derived from the requirements.




<<<<<<< HEAD
>>>>>>> Specification refactor
=======
%mainfile: ../../master.tex
\subsection{Structure}\label{sub:structure}

\cref{fig:structure} describe what classes are associated to what classes. This will be useful for the designers and system when designing and reasoning about the knowledge base \cref{sub:KB}.

\begin{figure}
  \label{fig:structure}
  \centering
  \begin{adjustbox}{max width=\textwidth}
    \includegraphics{Structure.pdf}
  \end{adjustbox}
  \caption{Structure of the problem domain}
\end{figure}



Elicited requirements: %Mangler argumentation

\begin{itemize}
\item The light in a users home will be turned off only when there is no need for it.
\item The user should be in doubt that the system will eventually turn off the light.
\item The should not be afraid that the system will suddenly turn off the light, whilst the user is need of it.
\item The light management should sought to be invisible to the user, light is turned on before a user can observe this behaviour.
\item Will learn usage patterns of the user, turn on light at specific time of day or at a specific light intensity.
\end{itemize}
>>>>>>> structure and usecases, splitted files
