From the system definition and "rich picture" of the problem domain that the system will be operating, we can extract some classes, \quote{A description of a collection of objects with the same structure, behaviour and attributes}, as described in \cref{OOAD}. This classes will be beneficial when figuring out what we need to keep track of in the problem domain, or what is relevant in our model of the problem domain.

Classes:
\begin{itemize}
\item Person - The user of the system
\item Action - An action the system should recognise as feedback
\item Pattern - The usage patter of a user, what is typical behaviour
\item Mistake - System did something wrong by the user
\item Sensor - The observational member of the system
\item Switch - A feedback source
\item Lamp - The light source on which system action
\item Light intensity - Luminance
<<<<<<< HEAD
<<<<<<< HEAD
<<<<<<< HEAD
\item Room - What room is the user in and its relations
=======
\item Room - What room is the user in
>>>>>>> structure and usecases, splitted files
=======
\item Room - What room is the user in and its relations
>>>>>>> delete trash
=======
\item Room - What room is the user in and its relations
>>>>>>> 9c8f5030b9365c258596d255a2898784b52995cb
\item Premise - The user home enviroment
\item Time - At what time, a pattern property
\end{itemize}
