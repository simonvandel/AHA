\section{Server design and implementation}
\subsection{Baum Welch}
\begin{frame}
	\frametitle{Baum Welch - Principle}
	\begin{itemize}
		\item N = number of hidden states in the model
		\item M = number of emission states in the model
		\item T = length of the observation sequence
		\item Problem: Given a sequence of observations $\mathcal{O}$ and the dimensions M and N, find the model $\theta = \{A, B, \pi\}$ that maximises the probability for $\mathcal{O}$. That is, the model that is most likely to have emitted the observations
		\item This can be solved via the Baum-Welch algorithm
	\end{itemize}
\end{frame}

\begin{frame}
	\frametitle{Baum Welch - Principle}
	\begin{itemize}
		\item Calculate $\alpha$/forwards and $\beta$/backwards
		\begin{itemize}
			\item $\alpha$ is the probability that the hidden state is $q$ at time $t$ given a partial observation sequence $\mathcal{O}$ from the start
			\item $\beta$ is the same as $\alpha$ but the observation sequence starts at the end to time $t$
		\end{itemize}
		\item Calculate di-gammas and gamma
		\begin{itemize}
			\item $\gamma_t(i,j)$ is the probability of being in hidden state $q_i$ at time $t$ and then transitioning to hidden state $q_j$ at time $t+1$
			\item $\gamma_t(i)$ is the probability of being in hidden state $q_i$ at time $t$
		\end{itemize}
		\item Recalculate $\theta = {A,B,\pi}$ from gamma and di-gamma
        \item Iterate over the model until $P(\mathcal{O}|\theta)$ does not rise
	\end{itemize}
\end{frame}

\begin{frame}
	\frametitle{Baum Welch - Example}
\begin{figure}[htbp]
\centering
\begin{tikzpicture}[->]
  %Hidden Nodes
  \node[circle, draw=black, minimum size=1cm] (x1) {$x_0$};
  \node[circle, draw=black, minimum size=1cm, node distance=3cm, right=of x1] (x2) {$x_1$};

  %Emission states
  \node[circle] (e11) [draw=black, minimum size=1cm, node distance=1.5cm, above=of x1] {$e_0$};
  \node[circle] (e12) [draw=black, minimum size=1cm, node distance=1.5cm, left=of x1] {$e_1$};
  \node[circle] (e21) [draw=black, minimum size=1cm, node distance=1.5cm, above=of x2] {$e_0$};
  \node[circle] (e22) [draw=black, minimum size=1cm, node distance=1.5cm, right=of x2] {$e_1$};

  \path
  %Emission lines
    (x1) edge[out=90,in=-90] node[rectangle,fill=white] {$b_{00}$} (e11)
    (x1) edge[out=180,in=0] node[rectangle,fill=white] {$b_{01}$} (e12)
    (x2) edge[out=90,in=-90] node[rectangle,fill=white] {$b_{10}$} (e21)
    (x2) edge[out=0,in=180] node[rectangle,fill=white] {$b_{11}$} (e22)

  %Transition lines
    (x1) edge[out=15,in=165] node[rectangle,fill=white] {$a_{01}$} (x2)
    (x2) edge[out=-165,in=-15] node[rectangle,fill=white] {$a_{10}$} (x1)

    (x1) edge [out=150,in=120,loop,looseness=15] node[rectangle,fill=white] {$a_{00}$} (x1)
    (x2) edge [out=60,in=30,loop,looseness=15] node[rectangle,fill=white] {$a_{11}$} (x2);
\end{tikzpicture}
\end{figure}
\begin{table}[htbp]
  \centering
  \begin{tabular}{| c | c | c |}
  	\hline
    $b_{ik} = P(e_k|x_i)$ & $e_0$ & $e_1$ \\ \hline
    $i = x_0$                 & 0.9   & 0.1   \\ \hline
    $i = x_1$                 & 0.2   & 0.8   \\ \hline
  \end{tabular}
  \begin{tabular}{| c | c | c |}
  	\hline
    $a_{ij} = P(x_j|x_i)$ & $j = x_0$ & $j = x_1$ \\ \hline
    $i = x_0$                 & 0.6   & 0.4   \\ \hline
    $i = x_1$                 & 0.9   & 0.1   \\ \hline
  \end{tabular}
  \\$\pi = P(x_i|t=0)$ =
  \begin{tabular}{| c | c |}
  	\hline
    $i = x_0$ & $i = x_1$ \\ \hline
    0.3   & 0.7   \\ \hline
  \end{tabular}
\end{table}
\end{frame}

\begin{frame}
	\frametitle{Baum Welch - Example}
\begin{figure}[htbp]
\centering
\begin{tikzpicture}[->]
  %Hidden Nodes
  \node[circle, draw=black, minimum size=1cm] (x1) {$x_0$};
  \node[circle, draw=black, minimum size=1cm, node distance=3cm, right=of x1] (x2) {$x_1$};

  %Emission states
  \node[circle] (e11) [draw=black, minimum size=1cm, node distance=1.5cm, above=of x1] {$e_0$};
  \node[circle] (e12) [draw=black, minimum size=1cm, node distance=1.5cm, left=of x1] {$e_1$};
  \node[circle] (e21) [draw=black, minimum size=1cm, node distance=1.5cm, above=of x2] {$e_0$};
  \node[circle] (e22) [draw=black, minimum size=1cm, node distance=1.5cm, right=of x2] {$e_1$};

  \path
  %Emission lines
    (x1) edge[out=90,in=-90] node[rectangle,fill=white] {$0.9$} (e11)
    (x1) edge[out=180,in=0] node[rectangle,fill=white] {$0.1$} (e12)
    (x2) edge[out=90,in=-90] node[rectangle,fill=white] {$0.2$} (e21)
    (x2) edge[out=0,in=180] node[rectangle,fill=white] {$0.8$} (e22)

  %Transition lines
    (x1) edge[out=15,in=165] node[rectangle,fill=white] {$0.4$} (x2)
    (x2) edge[out=-165,in=-15] node[rectangle,fill=white] {$0.9$} (x1)

    (x1) edge [out=150,in=120,loop,looseness=15] node[rectangle,fill=white] {$0.6$} (x1)
    (x2) edge [out=60,in=30,loop,looseness=15] node[rectangle,fill=white] {$0.1$} (x2);
\end{tikzpicture}
\end{figure}
  \begin{table}
  $\pi = P(x_i|t=0)$ =
  \begin{tabular}{| c | c |}
  	\hline
    $x_0$ & $x_1$ \\ \hline
    0.3   & 0.7   \\ \hline
  \end{tabular}
\end{table}
\end{frame}

\begin{frame}
	\frametitle{Baum Welch - Example}
  \begin{table}
  \centering
  $\mathcal{O} = \{\mathcal{O}_0 = e_0, \mathcal{O}_1 = e_1, \mathcal{O}_2 = e_0\}$\\
  \begin{tabular}{| c | c | c |}
  	\hline
    $b_{ik} = P(e_k|x_i)$ & $e_0$ & $e_1$ \\ \hline
    $i = x_0$                 & 0.9   & 0.1   \\ \hline
    $i = x_1$                 & 0.2   & 0.8   \\ \hline
  \end{tabular}
  \begin{tabular}{| c | c | c |}
  	\hline
    $a_{ij} = P(x_j|x_i)$ & $j = x_0$ & $j = x_1$ \\ \hline
    $i = x_0$                 & 0.6   & 0.4   \\ \hline
    $i = x_1$                 & 0.9   & 0.1   \\ \hline
  \end{tabular}
  \\$\pi = P(x_i|t=0)$ =
  \begin{tabular}{| c | c |}
  	\hline
    $i = x_0$ & $i = x_1$ \\ \hline
    0.3   & 0.7   \\ \hline
  \end{tabular}
  \begin{subfigure}{5.4cm}
  \centering
  \begin{framed}
  $\alpha_0(i) = \pi_i b_i(\mathcal{O}_0)$\\
  $\alpha_t(i) = \left(\sum\limits_{j=0}^{N-1} \alpha_{t-1}(j)a_{ji}\right)b_i(\mathcal{O}_t)$
  \end{framed}  
  \end{subfigure}
\end{table}
\end{frame}

\begin{frame}
	\frametitle{Baum Welch - Example}
  \begin{table}
  \centering
  $\mathcal{O} = \{\mathcal{O}_0 = e_0, \mathcal{O}_1 = e_1, \mathcal{O}_2 = e_0\}$\\
  \begin{tabular}{| c | c | c |}
  	\hline
    $b_{ik} = P(e_k|x_i)$ & $e_0$ & $e_1$ \\ \hline
    $i = x_0$                 & 0.9   & 0.1   \\ \hline
    $i = x_1$                 & 0.2   & 0.8   \\ \hline
  \end{tabular}
  \begin{tabular}{| c | c | c |}
  	\hline
    $a_{ij} = P(x_j|x_i)$ & $j = x_0$ & $j = x_1$ \\ \hline
    $i = x_0$                 & 0.6   & 0.4   \\ \hline
    $i = x_1$                 & 0.9   & 0.1   \\ \hline
  \end{tabular}
  \\$\pi = P(x_i|t=0)$ =
  \begin{tabular}{| c | c |}
  	\hline
    $i = x_0$ & $i = x_1$ \\ \hline
    0.3   & 0.7   \\ \hline
  \end{tabular}
  \begin{subfigure}{4.3cm}
  \begin{framed}
  $\alpha_0(0)=\pi_0 b_0(\mathcal{O}_0)$\\
  $\alpha_0(0)=0.3*0.9=0.27$\\
  $\alpha_0(1)=0.7*0.2=0.14$
  \end{framed}
  \end{subfigure}
  \begin{subfigure}{5.4cm}
  \centering
  \begin{framed}
  $\alpha_0(i) = \pi_i b_i(\mathcal{O}_0)$\\
  $\alpha_t(i) = \left(\sum\limits_{j=0}^{N-1} \alpha_{t-1}(j)a_{ji}\right)b_i(\mathcal{O}_t)$
  \end{framed}  
  \end{subfigure}
\end{table}
\end{frame}

\begin{frame}
	\frametitle{Baum Welch - Example}
  \begin{table}
  \centering
  $\mathcal{O} = \{\mathcal{O}_0 = e_0, \mathcal{O}_1 = e_1, \mathcal{O}_2 = e_0\}$\\
  \begin{tabular}{| c | c | c |}
  	\hline
    $b_{ik} = P(e_k|x_i)$ & $e_0$ & $e_1$ \\ \hline
    $i = x_0$                 & 0.9   & 0.1   \\ \hline
    $i = x_1$                 & 0.2   & 0.8   \\ \hline
  \end{tabular}
  \begin{tabular}{| c | c | c |}
  	\hline
    $a_{ij} = P(x_j|x_i)$ & $j = x_0$ & $j = x_1$ \\ \hline
    $i = x_0$                 & 0.6   & 0.4   \\ \hline
    $i = x_1$                 & 0.9   & 0.1   \\ \hline
  \end{tabular}
  \\$\pi = P(x_i|t=0)$ =
  \begin{tabular}{| c | c |}
  	\hline
    $i = x_0$ & $i = x_1$ \\ \hline
    0.3   & 0.7   \\ \hline
  \end{tabular}
  \begin{subfigure}{4.3cm}
  \begin{framed}
  $\alpha_0(0)=\pi_0 b_0(\mathcal{O}_0)$\\
  $\alpha_0(0)=0.3*0.9=0.27$\\
  $\alpha_0(1)=0.7*0.2=0.14$\\
  $\alpha_1(0)=\\(0.27*0.6+0.14*0.9)*0.1=0.0288$
  \end{framed}
  \end{subfigure}
  \begin{subfigure}{5.4cm}
  \centering
  \begin{framed}
  $\alpha_0(i) = \pi_i b_i(\mathcal{O}_0)$\\
  $\alpha_t(i) = \left(\sum\limits_{j=0}^{N-1} \alpha_{t-1}(j)a_{ji}\right)b_i(\mathcal{O}_t)$
  \end{framed}  
  \end{subfigure}
\end{table}
\end{frame}

\begin{frame}
	\frametitle{Baum Welch - Example}
  \begin{table}
  \centering
  $\mathcal{O} = \{\mathcal{O}_0 = e_0, \mathcal{O}_1 = e_1, \mathcal{O}_2 = e_0\}$\\
  \begin{tabular}{| c | c | c |}
  	\hline
    $b_{ik} = P(e_k|x_i)$ & $e_0$ & $e_1$ \\ \hline
    $i = x_0$                 & 0.9   & 0.1   \\ \hline
    $i = x_1$                 & 0.2   & 0.8   \\ \hline
  \end{tabular}
  \begin{tabular}{| c | c | c |}
  	\hline
    $a_{ij} = P(x_j|x_i)$ & $j = x_0$ & $j = x_1$ \\ \hline
    $i = x_0$                 & 0.6   & 0.4   \\ \hline
    $i = x_1$                 & 0.9   & 0.1   \\ \hline
  \end{tabular}
  \\$\pi = P(x_i|t=0)$ =
  \begin{tabular}{| c | c |}
  	\hline
    $i = x_0$ & $i = x_1$ \\ \hline
    0.3   & 0.7   \\ \hline
  \end{tabular}
  \begin{subfigure}{4.3cm}
  \begin{framed}
  $\alpha_0(0)=\pi_0 b_0(\mathcal{O}_0)$\\
  $\alpha_0(0)=0.3*0.9=0.27$\\
  $\alpha_0(1)=0.7*0.2=0.14$\\
  $\alpha_1(0)=\\(0.27*0.6+0.14*0.9)*0.1=0.0288$
  \end{framed}
  \end{subfigure}
  \begin{subfigure}{5.4cm}
  \centering
  \begin{framed}
  $\alpha_0(i) = \pi_i b_i(\mathcal{O}_0)$\\
  $\alpha_t(i) = \left(\sum\limits_{j=0}^{N-1} \alpha_{t-1}(j)a_{ji}\right)b_i(\mathcal{O}_t)$
  \end{framed}  
  \end{subfigure}
  $\alpha_t(i) = P(\mathcal{O}_0, \mathcal{O}_1, \dots ,\mathcal{O}_t , x_t = q_i | \theta) =$
  \begin{tabular}{| c | c | c | c |}
  	\hline
          & $t=e_0$ & $t=e_1$ & $t=e_0$ \\ \hline
    $x_0$ & 0.27    & 0.0288  & 0.094608 \\ \hline
    $x_1$ & 0.14    & 0.0976  & 0.004256 \\ \hline
  \end{tabular}
\end{table}
\end{frame}

\begin{frame}
	\frametitle{Baum Welch - Example}
  \begin{table}
  \centering
  $\mathcal{O} = \{\mathcal{O}_0 = e_0, \mathcal{O}_1 = e_1, \mathcal{O}_2 = e_0\}$\\
  \begin{tabular}{| c | c | c |}
  	\hline
    $b_{ik} = P(e_k|x_i)$ & $e_0$ & $e_1$ \\ \hline
    $i = x_0$                 & 0.9   & 0.1   \\ \hline
    $i = x_1$                 & 0.2   & 0.8   \\ \hline
  \end{tabular}
  \begin{tabular}{| c | c | c |}
  	\hline
    $a_{ij} = P(x_j|x_i)$ & $j = x_0$ & $j = x_1$ \\ \hline
    $i = x_0$                 & 0.6   & 0.4   \\ \hline
    $i = x_1$                 & 0.9   & 0.1   \\ \hline
  \end{tabular}
  \\$\pi = P(x_i|t=0)$ =
  \begin{tabular}{| c | c |}
  	\hline
    $i = x_0$ & $i = x_1$ \\ \hline
    0.3   & 0.7   \\ \hline
  \end{tabular}\\
  $\beta_{T-1}(i) = 1$
\end{table}
\end{frame}

\begin{frame}
	\frametitle{Baum Welch - Example}
  \begin{table}
  \centering
  $\mathcal{O} = \{\mathcal{O}_0 = e_0, \mathcal{O}_1 = e_1, \mathcal{O}_2 = e_0\}$\\
  \begin{tabular}{| c | c | c |}
  	\hline
    $b_{ik} = P(e_k|x_i)$ & $e_0$ & $e_1$ \\ \hline
    $i = x_0$                 & 0.9   & 0.1   \\ \hline
    $i = x_1$                 & 0.2   & 0.8   \\ \hline
  \end{tabular}
  \begin{tabular}{| c | c | c |}
  	\hline
    $a_{ij} = P(x_j|x_i)$ & $j = x_0$ & $j = x_1$ \\ \hline
    $i = x_0$                 & 0.6   & 0.4   \\ \hline
    $i = x_1$                 & 0.9   & 0.1   \\ \hline
  \end{tabular}
  \\$\pi = P(x_i|t=0)$ =
  \begin{tabular}{| c | c |}
  	\hline
    $i = x_0$ & $i = x_1$ \\ \hline
    0.3   & 0.7   \\ \hline
  \end{tabular}\\
  $\beta_{T-1}(i) = 1$\\
  $\beta_t(i) = \sum\limits_{j=0}^{N-1} a_{ij} b_j(\mathcal{O}_{t+1})\beta_{t+1}(j)$\\
  $\beta_1(1) = 0.9 * 0.9 * 1 + 0.1 * 0.2 * 1 = 0.83$
\end{table}
\end{frame}

\begin{frame}
	\frametitle{Baum Welch - Example}
  \begin{table}
  \centering
  $\mathcal{O} = \{\mathcal{O}_0 = e_0, \mathcal{O}_1 = e_1, \mathcal{O}_2 = e_0\}$\\
  \begin{tabular}{| c | c | c |}
  	\hline
    $b_{ik} = P(e_k|x_i)$ & $e_0$ & $e_1$ \\ \hline
    $i = x_0$                 & 0.9   & 0.1   \\ \hline
    $i = x_1$                 & 0.2   & 0.8   \\ \hline
  \end{tabular}
  \begin{tabular}{| c | c | c |}
  	\hline
    $a_{ij} = P(x_j|x_i)$ & $j = x_0$ & $j = x_1$ \\ \hline
    $i = x_0$                 & 0.6   & 0.4   \\ \hline
    $i = x_1$                 & 0.9   & 0.1   \\ \hline
  \end{tabular}
  \\$\pi = P(x_i|t=0)$ =
  \begin{tabular}{| c | c |}
  	\hline
    $i = x_0$ & $i = x_1$ \\ \hline
    0.3   & 0.7   \\ \hline
  \end{tabular}\\
  $\beta_{T-1}(i) = 1$\\
  $\beta_t(i) = \sum\limits_{j=0}^{N-1} a_{ij} b_j(\mathcal{O}_{t+1})\beta_{t+1}(j)$\\
  $\beta_1(1) = 0.9 * 0.9 * 1 + 0.1 * 0.2 * 1 = 0.83$
  $\beta_t(i) = \sum\limits_{j=0}^{N-1} a_{ij} b_j(\mathcal{O}_{t+1})\beta_{t+1}(j) =$
  \begin{tabular}{| c | c | c | c |}
  	\hline
          & $t=e_0$ & $t=e_1$ & $t=e_0$ \\ \hline
    $x_0$ & 0.3028  & 0.62    & 1       \\ \hline
    $x_1$ & 0.1222  & 0.83    & 1       \\ \hline
  \end{tabular}\\
\end{table}
\end{frame}

\begin{frame}
	\frametitle{Baum Welch - Example}
  \begin{table}
  \centering
  $\mathcal{O} = \{\mathcal{O}_0 = e_0, \mathcal{O}_1 = e_1, \mathcal{O}_2 = e_0\}$\\
  \begin{tabular}{| c | c | c |}
  	\hline
    $b_{ik} = P(e_k|x_i)$ & $e_0$ & $e_1$ \\ \hline
    $i = x_0$                 & 0.9   & 0.1   \\ \hline
    $i = x_1$                 & 0.2   & 0.8   \\ \hline
  \end{tabular}
  \begin{tabular}{| c | c | c |}
  	\hline
    $a_{ij} = P(x_j|x_i)$ & $j = x_0$ & $j = x_1$ \\ \hline
    $i = x_0$                 & 0.6   & 0.4   \\ \hline
    $i = x_1$                 & 0.9   & 0.1   \\ \hline
  \end{tabular}
  \\$\pi = P(x_i|t=0)$ =
  \begin{tabular}{| c | c |}
  	\hline
    $i = x_0$ & $i = x_1$ \\ \hline
    0.3   & 0.7   \\ \hline
  \end{tabular}\\  
  $\alpha_t(i) = P(\mathcal{O}_0, \mathcal{O}_1, \dots ,\mathcal{O}_t , x_t = q_i | \theta) =$
  \begin{tabular}{| c | c | c | c |}
  	\hline
          & $t=e_0$ & $t=e_1$ & $t=e_0$ \\ \hline
    $x_0$ & 0.27    & 0.0288  & 0.094608 \\ \hline
    $x_1$ & 0.14    & 0.0976  & 0.004256 \\ \hline
  \end{tabular}\\
  $\beta_t(i) = \sum\limits_{j=0}^{N-1} a_{ij} b_j(\mathcal{O}_{t+1})\beta_{t+1}(j) =$
  \begin{tabular}{| c | c | c | c |}
  	\hline
          & $t=e_0$ & $t=e_1$ & $t=e_0$ \\ \hline
    $x_0$ & 0.3028  & 0.62    & 1       \\ \hline
    $x_1$ & 0.1222  & 0.83    & 1       \\ \hline
  \end{tabular}\\
\end{table}
\end{frame}

\begin{frame}
	\frametitle{Baum Welch - Example}
  \begin{table}
  $\alpha_t(i) = P(\mathcal{O}_0, \mathcal{O}_1, \dots ,\mathcal{O}_t , x_t = q_i | \theta) =$
  \begin{tabular}{| c | c | c | c |}
  	\hline
          & $t=e_0$ & $t=e_1$ & $t=e_0$ \\ \hline
    $x_0$ & 0.27    & 0.0288  & 0.0946 \\ \hline
    $x_1$ & 0.14    & 0.0976  & 0.0043 \\ \hline
  \end{tabular}\\
  $\beta_t(i) = \sum\limits_{j=0}^{N-1} a_{ij} b_j(\mathcal{O}_{t+1})\beta_{t+1}(j) =$
  \begin{tabular}{| c | c | c | c |}
  	\hline
          & $t=e_0$ & $t=e_1$ & $t=e_0$ \\ \hline
    $x_0$ & 0.3028  & 0.62    & 1       \\ \hline
    $x_1$ & 0.1222  & 0.83    & 1       \\ \hline
  \end{tabular}\\
  $\gamma_t(i) = P(x_t = q_i|\mathcal{O},\theta)$\\
  $\gamma_t(i) = \frac{\alpha_t(i)\beta_t(i)}{P(\mathcal{O}|\theta)} = \frac{\alpha_t(i)\beta_t(i)}{\sum\limits_{j=0}^{N-1} \alpha_{T-1}(j)}$
\end{table}
\end{frame}

\begin{frame}
	\frametitle{Baum Welch - Example}
  \begin{table}
  $\alpha_t(i) = P(\mathcal{O}_0, \mathcal{O}_1, \dots ,\mathcal{O}_t , x_t = q_i | \theta) =$
  \begin{tabular}{| c | c | c | c |}
  	\hline
          & $t=e_0$ & $t=e_1$ & $t=e_0$ \\ \hline
    $x_0$ & 0.27    & 0.0288  & 0.0946 \\ \hline
    $x_1$ & 0.14    & 0.0976  & 0.0043 \\ \hline
  \end{tabular}\\
  $\beta_t(i) = \sum\limits_{j=0}^{N-1} a_{ij} b_j(\mathcal{O}_{t+1})\beta_{t+1}(j) =$
  \begin{tabular}{| c | c | c | c |}
  	\hline
          & $t=e_0$ & $t=e_1$ & $t=e_0$ \\ \hline
    $x_0$ & 0.3028  & 0.62    & 1       \\ \hline
    $x_1$ & 0.1222  & 0.83    & 1       \\ \hline
  \end{tabular}\\
  $\gamma_t(i) = P(x_t = q_i|\mathcal{O},\theta)$\\
  $\gamma_t(i) = \frac{\alpha_t(i)\beta_t(i)}{P(\mathcal{O}|\theta)} = \frac{\alpha_t(i)\beta_t(i)}{\sum\limits_{j=0}^{N-1} \alpha_{T-1}(j)}$\\
  $\gamma_1(0) = \frac{0.0288 * 0.62}{0.0946 + 0.0043} = 0.181$\\
\end{table}
\end{frame}

\begin{frame}
	\frametitle{Baum Welch - Example}
  \begin{table}
  $\alpha_t(i) = P(\mathcal{O}_0, \mathcal{O}_1, \dots ,\mathcal{O}_t , x_t = q_i | \theta) =$
  \begin{tabular}{| c | c | c | c |}
  	\hline
          & $t=e_0$ & $t=e_1$ & $t=e_0$ \\ \hline
    $x_0$ & 0.27    & 0.0288  & 0.0946 \\ \hline
    $x_1$ & 0.14    & 0.0976  & 0.0043 \\ \hline
  \end{tabular}\\
  $\beta_t(i) = \sum\limits_{j=0}^{N-1} a_{ij} b_j(\mathcal{O}_{t+1})\beta_{t+1}(j) =$
  \begin{tabular}{| c | c | c | c |}
  	\hline
          & $t=e_0$ & $t=e_1$ & $t=e_0$ \\ \hline
    $x_0$ & 0.3028  & 0.62    & 1       \\ \hline
    $x_1$ & 0.1222  & 0.83    & 1       \\ \hline
  \end{tabular}\\
  $\gamma_t(i) = P(x_t = q_i|\mathcal{O},\theta)$\\
  $\gamma_t(i) = \frac{\alpha_t(i)\beta_t(i)}{P(\mathcal{O}|\theta)} = \frac{\alpha_t(i)\beta_t(i)}{\sum\limits_{j=0}^{N-1} \alpha_{T-1}(j)}$\\
  $\gamma_1(0) = \frac{0.0288 * 0.62}{0.0946 + 0.0043} = 0.181$\\
  $\gamma_t(i) = P(x_t = q_i|\mathcal{O},\theta) =$
  \begin{tabular}{| c | c | c | c |}
  	\hline
          & $t=e_0$ & $t=e_1$ & $t=e_0$ \\ \hline
    $x_0$ & 0.827  & 0.181  & 0.957   \\ \hline
    $x_1$ & 0.173  & 0.819  & 0.043   \\ \hline
  \end{tabular}\\
\end{table}
\end{frame}

\begin{frame}
	\frametitle{Baum Welch - Example}
  \begin{table}
  $\alpha_t(i) = P(\mathcal{O}_0, \mathcal{O}_1, \dots ,\mathcal{O}_t , x_t = q_i | \theta) =$
  \begin{tabular}{| c | c | c | c |}
  	\hline
          & $t=e_0$ & $t=e_1$ & $t=e_0$ \\ \hline
    $x_0$ & 0.27    & 0.0288  & 0.0946 \\ \hline
    $x_1$ & 0.14    & 0.0976  & 0.0043 \\ \hline
  \end{tabular}\\
  $\beta_t(i) = \sum\limits_{j=0}^{N-1} a_{ij} b_j(\mathcal{O}_{t+1})\beta_{t+1}(j) =$
  \begin{tabular}{| c | c | c | c |}
  	\hline
          & $t=e_0$ & $t=e_1$ & $t=e_0$ \\ \hline
    $x_0$ & 0.2748  & 0.52    & 1       \\ \hline
    $x_1$ & 0.1782  & 0.73    & 1       \\ \hline
  \end{tabular}\\
  $y_t(i,j) = P(x_t = q_i, x_{t+1} = q_j | \mathcal{O},\theta)$
  $y_t(i,j) = \frac{\alpha_t(i) a_{ij} b_j (\mathcal{O}_{t+1})\beta_{t+1}(j)}{P(\mathcal{O} | \theta)} = \frac{\alpha_t(i) a_{ij} b_j (\mathcal{O}_{t+1})\beta_{t+1}(j)}{\sum\limits_{i=0}^{N-1} \alpha_{T-1}(i)} =$
    \begin{tabular}{| c | c | c | c | c |}
  	\hline
  	      & \multicolumn{2}{| c |}{$t=0$} & \multicolumn{2}{| c |}{$t=1$} \\ \hline
          & $j=0$ & $j=1$ & $j=0$ & $j=1$ \\ \hline
    $i=0$ & 0.1016  & 0.7254 & 0.1573 & 0.0233 \\ \hline
    $i=1$ & 0.079  & 0.094 & 0.7996 & 0.0197 \\ \hline
  \end{tabular}\\
\end{table}
\end{frame}

\begin{frame}
	\frametitle{Baum Welch - Example}
  \begin{table}
  \begin{tabular}{| c | c | c | c |}
  	\hline
    $\gamma_t(i)$      & $t=e_0$ & $t=e_1$ & $t=e_0$ \\ \hline
    $x_0$ & 0.827  & 0.181  & 0.957   \\ \hline
    $x_1$ & 0.173  & 0.819  & 0.043   \\ \hline
  \end{tabular}\\
    \begin{tabular}{| c | c | c | c | c |}
  	\hline
  	      & \multicolumn{2}{| c |}{$t=0$} & \multicolumn{2}{| c |}{$t=1$} \\ \hline
    $\gamma_t(ij)$      & $j=0$ & $j=1$ & $j=0$ & $j=1$ \\ \hline
    $i=0$ & 0.1016  & 0.7254 & 0.1573 & 0.0233 \\ \hline
    $i=1$ & 0.079  & 0.094 & 0.7996 & 0.0197 \\ \hline
  \end{tabular}\\
    $\pi_i = \gamma_0(i)$\\
    $a_{ij} = \frac{\sum\limits_{t=0}^{T-2} \gamma_t (i,j)}{\sum\limits_{t=0}^{T-2} \gamma_t(i)}$\\
    $b_j(k) = \frac{\sum\limits_{t \in \{0,1,\dots,T-1\} and \mathcal{O}_t=k} \gamma_t (j)}{\sum\limits_{t=0}^{T-1} \gamma_t(i)}$
\end{table}
\end{frame}

\begin{frame}
	\frametitle{Baum Welch - Example}
  \begin{table}
  \begin{tabular}{| c | c | c | c |}
  	\hline
    $\gamma_t(i)$      & $t=e_0$ & $t=e_1$ & $t=e_0$ \\ \hline
    $x_0$ & 0.827  & 0.181  & 0.957   \\ \hline
    $x_1$ & 0.173  & 0.819  & 0.043   \\ \hline
  \end{tabular}\\
    \begin{tabular}{| c | c | c | c | c |}
  	\hline
  	      & \multicolumn{2}{| c |}{$t=0$} & \multicolumn{2}{| c |}{$t=1$} \\ \hline
    $\gamma_t(ij)$      & $j=0$ & $j=1$ & $j=0$ & $j=1$ \\ \hline
    $i=0$ & 0.1016  & 0.7254 & 0.1573 & 0.0233 \\ \hline
    $i=1$ & 0.079  & 0.094 & 0.7996 & 0.0197 \\ \hline
  \end{tabular}\\
    $\pi_i = \gamma_0(i)$\\
    $\pi = P(x_i|t=0)$ =
  \begin{tabular}{| c | c |}
  	\hline
    $i = x_0$ & $i = x_1$ \\ \hline
    0.827   & 0.173   \\ \hline
  \end{tabular}
\end{table}
\end{frame}

\begin{frame}
	\frametitle{Baum Welch - Example}
  \begin{table}
  \begin{tabular}{| c | c | c | c |}
  	\hline
    $\gamma_t(i)$      & $t=e_0$ & $t=e_1$ & $t=e_0$ \\ \hline
    $x_0$ & 0.827  & 0.181  & 0.957   \\ \hline
    $x_1$ & 0.173  & 0.819  & 0.043   \\ \hline
  \end{tabular}\\
    \begin{tabular}{| c | c | c | c | c |}
  	\hline
  	      & \multicolumn{2}{| c |}{$t=0$} & \multicolumn{2}{| c |}{$t=1$} \\ \hline
    $\gamma_t(ij)$      & $j=0$ & $j=1$ & $j=0$ & $j=1$ \\ \hline
    $i=0$ & 0.1016  & 0.7254 & 0.1573 & 0.0233 \\ \hline
    $i=1$ & 0.079  & 0.094 & 0.7996 & 0.0197 \\ \hline
  \end{tabular}\\
    $a_{ij} = \frac{\sum\limits_{t=0}^{T-2} \gamma_t (i,j)}{\sum\limits_{t=0}^{T-2} \gamma_t(i)}$\\
    $a_{10} = \frac{0.079 + 0.7996}{0.173 + 0.819} = 0.885$\\
  \begin{tabular}{| c | c | c |}
  	\hline
    $a_{ij}$ & $j = x_0$ & $j = x_1$ \\ \hline
    $i = x_0$                 & 0.257   & 0.743   \\ \hline
    $i = x_1$                 & 0.885   & 0.115   \\ \hline
  \end{tabular}
\end{table}
\end{frame}

\begin{frame}
	\frametitle{Baum Welch - Example}
  \begin{table}
  \begin{tabular}{| c | c | c | c |}
  	\hline
    $\gamma_t(i)$      & $t=e_0$ & $t=e_1$ & $t=e_0$ \\ \hline
    $x_0$ & 0.827  & 0.181  & 0.957   \\ \hline
    $x_1$ & 0.173  & 0.819  & 0.043   \\ \hline
  \end{tabular}\\
    \begin{tabular}{| c | c | c | c | c |}
  	\hline
  	      & \multicolumn{2}{| c |}{$t=0$} & \multicolumn{2}{| c |}{$t=1$} \\ \hline
    $\gamma_t(ij)$      & $j=0$ & $j=1$ & $j=0$ & $j=1$ \\ \hline
    $i=0$ & 0.1016  & 0.7254 & 0.1573 & 0.0233 \\ \hline
    $i=1$ & 0.079  & 0.094 & 0.7996 & 0.0197 \\ \hline
  \end{tabular}\\
    $b_j(k) = \frac{\sum\limits_{t \in \{0,1,\dots,T-1\} and \mathcal{O}_t=k} \gamma_t (j)}{\sum\limits_{t=0}^{T-1} \gamma_t(i)}$\\
    $b_0(0) = \frac{0.827 + 0.957}{0.827 + 0.181 + 0.957} = 0.908$\\
  \begin{tabular}{| c | c | c |}
  	\hline
    $b_{ik}$ & $e_0$ & $e_1$ \\ \hline
    $i = x_0$                 & 0.908   & 0.092   \\ \hline
    $i = x_1$                 & 0.209   & 0.791   \\ \hline
  \end{tabular}
\end{table}
\end{frame}

\begin{frame}
	\frametitle{Baum Welch - Example}
  \begin{table}
  \begin{tabular}{| c | c | c |}
  	\hline
    $a_{ij}$ & $j = x_0$ & $j = x_1$ \\ \hline
    $i = x_0$                 & 0.257   & 0.743   \\ \hline
    $i = x_1$                 & 0.885   & 0.115   \\ \hline
  \end{tabular}
  \begin{tabular}{| c | c | c |}
  	\hline
    $b_{ik}$ & $e_0$ & $e_1$ \\ \hline
    $i = x_0$                 & 0.908   & 0.092   \\ \hline
    $i = x_1$                 & 0.209   & 0.791   \\ \hline
  \end{tabular}
  $\pi = P(x_i|t=0) = $
  \begin{tabular}{| c | c |}
  	\hline
    $i = x_0$ & $i = x_1$ \\ \hline
    0.827   & 0.173   \\ \hline
  \end{tabular}
  \begin{figure}[htbp]
\centering
\begin{tikzpicture}[->]
  %Hidden Nodes
  \node[circle, draw=black, minimum size=1cm] (x1) {$x_0$};
  \node[circle, draw=black, minimum size=1cm, node distance=3cm, right=of x1] (x2) {$x_1$};

  %Emission states
  \node[circle] (e11) [draw=black, minimum size=1cm, node distance=1.5cm, above=of x1] {$e_0$};
  \node[circle] (e12) [draw=black, minimum size=1cm, node distance=1.5cm, left=of x1] {$e_1$};
  \node[circle] (e21) [draw=black, minimum size=1cm, node distance=1.5cm, above=of x2] {$e_0$};
  \node[circle] (e22) [draw=black, minimum size=1cm, node distance=1.5cm, right=of x2] {$e_1$};

  \path
  %Emission lines
    (x1) edge[out=90,in=-90] node[rectangle,fill=white] {0.908} (e11)
    (x1) edge[out=180,in=0] node[rectangle,fill=white] {0.092} (e12)
    (x2) edge[out=90,in=-90] node[rectangle,fill=white] {0.209} (e21)
    (x2) edge[out=0,in=180] node[rectangle,fill=white] {0.791} (e22)

  %Transition lines
    (x1) edge[out=15,in=165] node[rectangle,fill=white] {0.743} (x2)
    (x2) edge[out=-165,in=-15] node[rectangle,fill=white] {0.885} (x1)

    (x1) edge [out=150,in=120,loop,looseness=15] node[rectangle,fill=white] {0.257} (x1)
    (x2) edge [out=60,in=30,loop,looseness=15] node[rectangle,fill=white] {0.115} (x2);
\end{tikzpicture}
\end{figure}
\end{table}
\end{frame}

\begin{frame}
	\frametitle{Baum Welch - Example}
  \begin{table}
  $\pi = P(x_i|t=0) = $
  \begin{tabular}{| c | c |}
  	\hline
    $i = x_0$ & $i = x_1$ \\ \hline
    0.827   & 0.173   \\ \hline
  \end{tabular}
    \begin{figure}[htbp]
\begin{tikzpicture}[->]
  %Hidden Nodes
  \node[circle, draw=black, minimum size=1cm] (x1) {$x_0$};
  \node[circle, draw=black, minimum size=1cm, node distance=3cm, right=of x1] (x2) {$x_1$};

  %Emission states
  \node[circle] (e11) [draw=black, minimum size=1cm, node distance=1.5cm, above=of x1] {$e_0$};
  \node[circle] (e12) [draw=black, minimum size=1cm, node distance=1.5cm, left=of x1] {$e_1$};
  \node[circle] (e21) [draw=black, minimum size=1cm, node distance=1.5cm, above=of x2] {$e_0$};
  \node[circle] (e22) [draw=black, minimum size=1cm, node distance=1.5cm, right=of x2] {$e_1$};

  \path
  %Emission lines
    (x1) edge[out=90,in=-90] node[rectangle,fill=white] {0.908} (e11)
    (x1) edge[out=180,in=0] node[rectangle,fill=white] {0.092} (e12)
    (x2) edge[out=90,in=-90] node[rectangle,fill=white] {0.209} (e21)
    (x2) edge[out=0,in=180] node[rectangle,fill=white] {0.791} (e22)

  %Transition lines
    (x1) edge[out=15,in=165] node[rectangle,fill=white] {0.743} (x2)
    (x2) edge[out=-165,in=-15] node[rectangle,fill=white] {0.885} (x1)

    (x1) edge [out=150,in=120,loop,looseness=15] node[rectangle,fill=white] {0.257} (x1)
    (x2) edge [out=60,in=30,loop,looseness=15] node[rectangle,fill=white] {0.115} (x2);
\end{tikzpicture}
$\mathcal{O} = \{e_0, e_1, e_0\}$\\
$P(\mathcal{O}|\theta) = 0.827 * 0.908 * 0.743 * 0.791 * 0.885 * 0.908 = 0.3546$
$P(\mathcal{O}|\theta_{old}) = 0.3 * 0.9 * 0.4 * 0.8 * 0.9 * 0.9 = 0.07$
\end{figure}
\end{table}
\end{frame}