\section{Server implementation - Learner}
\lstset{language=HMMLanguage,numbers=left,breaklines=true,numbersep=0pt}
\subsection{Baum Welch}
\begin{frame}
	\frametitle{Baum Welch - Principle}
	\begin{itemize}
		\item N = number of hidden states in the model
		\item M = number of emission states in the model
		\item T = length of the observation sequence
		\item Problem: Given a sequence of observations $\mathcal{O}$ and the dimensions M and N, find the model $\theta = \{A, B, \pi\}$ that maximises the probability for $\mathcal{O}$. That is, the model that is most likely to have emitted the observations
		\item This can be solved via the Baum-Welch algorithm
	\end{itemize}
\end{frame}

\begin{frame}
	\frametitle{Baum Welch - Principle}
	\begin{itemize}
		\item Calculate $\alpha$/forwards and $\beta$/backwards
		\begin{itemize}
			\item $\alpha$ is the probability that the hidden state is $q$ at time $t$ given a partial observation sequence $\mathcal{O}$ from the start
			\item $\beta$ is the same as $\alpha$ but the observation sequence starts at the end to time $t$
		\end{itemize}
		\item Calculate di-gammas and gamma
		\begin{itemize}
			\item $\gamma_t(i,j)$ is the probability of being in hidden state $q_i$ at time $t$ and then transitioning to hidden state $q_j$ at time $t+1$
			\item $\gamma_t(i)$ is the probability of being in hidden state $q_i$ at time $t$
		\end{itemize}
		\item Recalculate $\theta = {A,B,\pi}$ from gamma and di-gamma
	\end{itemize}
\end{frame}

\begin{frame}
	\frametitle{Baum Welch - Calculate $\alpha$ and $\beta$}
	\begin{itemize}
		\item $\alpha$ - Can be computed recursively
		\begin{itemize}
			\item $\alpha_t(i) = P(\mathcal{O}_0, \mathcal{O}_1, \dots ,\mathcal{O}_t , x_t = q_i | \theta)$
			\item $\alpha_0(i) = \pi_i b_i(\mathcal{O}_0)$ for $i = 0, 1, \dots, N-1$
			\item $\alpha_t(i) = \left(\sum\limits_{j=0}^{N-1} \alpha_{t-1}(j)a_{ji}\right)b_i(\mathcal{O}_t)$
			\item $P(\mathcal{O}|\theta) = \sum\limits_{i=0}^{N-1} \alpha_{T-1}(i)$
		\end{itemize}
	\end{itemize}
\end{frame}

\begin{frame}
	\frametitle{Baum Welch - Calculate $\alpha$ and $\beta$}
	\begin{itemize}
		\item $\beta$ - Can also be calculated recursively
		\begin{itemize}
			\item $\beta_t(i) = P(\mathcal{O}_{t+1}, \mathcal{O}_{t+2}, \dots, \mathcal{O}_{T-1} | x_t = q_i, \theta)$
			\item $\beta_{T-1}(i) = 1$ for $i = 0, 1, \dots, N-1$
			\item $\beta_t(i) = \sum\limits_{j=0}^{N-1} a_{ij} b_j(\mathcal{O}_{t+1}) \beta_{t+1}(j)$
		\end{itemize}
	\end{itemize}
\end{frame}

\begin{frame}[fragile]
	\frametitle{Baum Welch - Calculate $\alpha$ and $\beta$}
	\begin{lstlisting}    
 // beta T-1 (i) = 1, scaled by c T-1
 for (HiddenState i : mapWarden.iterateHiddenStates()){
   Observation lastObservation = mapWarden.lastObservation();
   double value = mapWarden.getScalingFactor(mapWarden.getNumObservations() - 1); // T - 1
   setEntry(lastObservation, i, value);
 }
	\end{lstlisting}
\end{frame}

\begin{frame}[fragile]
	\frametitle{Baum Welch - Calculate $\alpha$ and $\beta$}
	\begin{lstlisting}   

 //beta pass
 for (int t = mapWarden.getNumObservations()-2; t >= 0; t--){
   Observation tObservation = mapWarden.observationIndexToObservation(t);
   for (HiddenState i : mapWarden.iterateHiddenStates()){
     setEntry(tObservation, i, 0);
     for (HiddenState j : mapWarden.iterateHiddenStates()){
       Observation nextObservation = mapWarden.observationIndexToObservation(t + 1);
       EmissionState nextEmission = mapWarden.observationToEmission(nextObservation);
       double value = getEntry(tObservation, i) + oldModel.getTransitionMatrix().getEntry(i, j) * oldModel.getEmissionMatrix().getEntry(j, nextEmission) * getEntry(nextObservation, j);
       setEntry(tObservation, i, value);
     }
     //scale beta t (i) with the same scale factor as alpha t (i)
     double value = mapWarden.getScalingFactor(t) * getEntry(tObservation, i);
     setEntry(tObservation, i, value);
   }
   int i = 0;
 }
	\end{lstlisting}
\end{frame}

\begin{frame}
	\frametitle{Baum Welch - Calculate di-$\gamma$ and $\gamma$}
	\begin{itemize}
		\item Di-gammas
		\begin{itemize}
			\item $y_t(i,j) = P(x_t = q_i, x_{t+1} = q_j | \mathcal{O},\theta)$
			\item $y_t(i,j) = \frac{\alpha_t(i) a_{ij} b_j (\mathcal{O}_{t+1})\beta_{t+1}(j)}{P(\mathcal{O} | \theta)} = \frac{\alpha_t(i) a_{ij} b_j (\mathcal{O}_{t+1})\beta_{t+1}(j)}{\sum\limits_{i=0}^{N-1} \alpha_{T-1}(i)}$
		\end{itemize}
	\end{itemize}
\end{frame}

\begin{frame}
	\frametitle{Baum Welch - Calculate di-$\gamma$ and $\gamma$}
	\begin{itemize}
		\item Gamma
		\begin{itemize}
			\item $\gamma_t(i) = P(x_t = q_i|\mathcal{O},\theta)$
			\item $\gamma_t(i) = \frac{\alpha_t(i)\beta_t(i)}{P(\mathcal{O}|\theta)} = \frac{\alpha_t(i)\beta_t(i)}{\sum\limits_{i=0}^{N-1} \alpha_{T-1}(i)}$
			\item or
			\item $ \sum\limits_{j=0}^{N-1} \gamma_t(i,j)$
		\end{itemize}
	\end{itemize}
\end{frame}

\begin{frame}
	\frametitle{Baum Welch - Recalculate $\theta$}
	\begin{itemize}
		\item Recalculate initial probabilities
		\begin{itemize}
			\item $\pi_i = \gamma_0(i)$
		\end{itemize}
		\item Recalculate transition matrix
		\begin{itemize}
			\item $a_{ij} = \frac{\sum\limits_{i=0}^{N-2} \gamma_t (j)}{\sum\limits_{i=0}^{N-2} \gamma_t(i)}$
		\end{itemize}
		\item Recalculate emission matrix
		\begin{itemize}
			\item $b_j(k) = \frac{\sum\limits_{t \in \{0,1,\dots,T-1\} and \mathcal{O}_t=k} \gamma_t (j)}{\sum\limits_{i=0}^{N-1} \gamma_t(i)}$
		\end{itemize}
	\end{itemize}
\end{frame}