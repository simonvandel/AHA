\begin{frame}
\frametitle{Overview}
\begin{itemize}
\item Veterbi
\item Communication
\item Normaliser
\item Sampler
\end{itemize}
\end{frame}
\subsection{Viterbi}
\begin{frame}
	\frametitle{Viterbi}
     \begin{itemize}
       \item Calculate viterbi path
     \end{itemize}
\begin{figure}[htbp]
\centering
\begin{tikzpicture}[scale=0.8, every node/.style={transform shape}]
  %Hidden Nodes
  \node[ellipse] (h1)   [draw=black, minimum width=1cm, minimum height=.75cm] {$x_1$};
  \node[ellipse] (h2)   [draw=black, minimum width=1cm, minimum height=.75cm, right=of h1] {$x_2$};
  \node          (dots) [draw=none,  minimum width=1cm, minimum height=.75cm, right=of h2] {\LARGE \dots};
  \node[ellipse] (htp1) [draw=black, minimum width=1cm, minimum height=.75cm, right=of dots] {$x_{t-1}$};
  \node[ellipse] (ht)   [draw=black, minimum width=1cm, minimum height=.75cm, right=of htp1] {$x_t$};
  \node[ellipse] (htf1) [draw=black, minimum width=1cm, minimum height=.75cm, right=of ht] {$x_{t+1}$};

  %Emission states
  \node[ellipse] (e1)   [draw=black, minimum width=1cm, minimum height=.75cm, below=of h1] {$o_1$};
  \node[ellipse] (e2)   [draw=black, minimum width=1cm, minimum height=.75cm, below=of h2] {$o_2$};
  \node[ellipse] (etp1) [draw=black, minimum width=1cm, minimum height=.75cm, below=of htp1] {$o_{t-1}$};
  \node[ellipse] (et)   [draw=black, minimum width=1cm, minimum height=.75cm, below=of ht] {$o_t$};
  \node[ellipse] (etf1) [draw=black, minimum width=1cm, minimum height=.75cm, below=of htf1] {$o_{t+1}$};

  %Emission lines
  \draw [->, to path={-| (\tikztotarget)}] (h1) edge[out=-90,in=90] (e1);
  \draw [->, to path={-| (\tikztotarget)}] (h2) edge[out=-90,in=90] (e2);
  \draw [->, to path={-| (\tikztotarget)}] (htp1) edge[out=-90,in=90] (etp1);
  \draw [->, to path={-| (\tikztotarget)}] (ht) edge[out=-90,in=90] (et);
  \draw [->, to path={-| (\tikztotarget)}] (htf1) edge[out=-90,in=90] (etf1);

  %1. order lines
  \draw [->, to path={-| (\tikztotarget)}] (h1) edge[out=0,in=180] (h2);
  \draw [->, to path={-| (\tikztotarget)}] (h2) edge[out=0,in=180] (dots);
  \draw [->, to path={-| (\tikztotarget)}] (dots) edge[out=0,in=180] (htp1);
  \draw [->, to path={-| (\tikztotarget)}] (htp1) edge[out=0,in=180] (ht);
  \draw [->, to path={-| (\tikztotarget)}] (ht) edge[out=0,in=180] (htf1);
\end{tikzpicture}
\caption[Absolute and relative time trellis diagram for an HMM]{An absolute and relative time trellis diagram for a 1st order HMM. The $x_{1}$, $x_{2}$ are absolute times where $x_{t-1}$, $x_{t}$ are relative, since the time depend on t}
\end{figure}\label{fig:4thMarkovModel}

\end{frame}
\section{Server implementation - Normaliser, sampler, java com}
\subsection{Normaliser, sampler, java com}
\begin{frame}
	\frametitle{Communication, Normaliser and Sampler}
	\begin{itemize}
		\item The Model of the System
    \end{itemize}
	\begin{figure}[htbp]
  	\centering
		\begin{tikzpicture}[scale=0.6, every node/.style={transform shape}]
			%Nodes
			\node[squarednode] (embeddedSubsystem) {Embedded subsystem};
			\node[squarednode] (communication) [right=of embeddedSubsystem, xshift=5em] {Communication};
			\node[squarednode] (normaliser) [right=of communication, xshift=5em] {Normaliser};
			\node[squarednode] (reasoner) [below=of embeddedSubsystem, yshift=-3em] {Reasoner};
			\node[squarednode] (sampler) [below=of normaliser, yshift=-3em] {Sampler};
			\node[squarednode] (learner) [below=of reasoner, yshift=-2em] {Learner};

			%Lines
			\path[-triangle 90]
			(embeddedSubsystem.east) edge[transform canvas={yshift=-0.5em}] node[anchor=north] {sensor data} (communication.west)
			(communication.west) edge[transform canvas={yshift=0.5em}] node[anchor=south] {actions} (embeddedSubsystem.east)
			(communication.east) edge node[anchor=south] {snapshot, time} (normaliser)
			(normaliser.south) edge node[anchor=east,text width=6em] {normalised snapshot, time} (sampler.north)
			(sampler) edge[out=south, in=east] node[anchor=north,text width=6em] {scope, time, actions}  (learner)
			(learner) edge[out=north, in=south, transform canvas={xshift=-1em}] node[anchor=east] {model} (reasoner)
			(sampler.west) edge node[anchor=south, text width=6em] {scope, time, actions} (reasoner.east)
			(reasoner) edge[out=north,in=south] node[anchor=north] {actions} (communication)
			(reasoner) edge[out=south,in=north, transform canvas={xshift=1em}] node[anchor=west] {feedback} (learner);
		\end{tikzpicture}
  \caption{Architecture of the modules in the learning subsystem}
  \end{figure}
  \end{frame}
  \begin{frame}[fragile]
	\frametitle{Veterbi - Example}
\begin{figure}[htbp]
\centering
\begin{tikzpicture}[->]
  %Hidden Nodes
  \node[circle, draw=black, minimum size=1cm] (x1) {$x_0$};
  \node[circle, draw=black, minimum size=1cm, node distance=3cm, right=of x1] (x2) {$x_1$};

  %Emission states
  \node[circle] (e11) [draw=black, minimum size=1cm, node distance=1.5cm, above=of x1] {$e_0$};
  \node[circle] (e12) [draw=black, minimum size=1cm, node distance=1.5cm, left=of x1] {$e_1$};
  \node[circle] (e21) [draw=black, minimum size=1cm, node distance=1.5cm, above=of x2] {$e_0$};
  \node[circle] (e22) [draw=black, minimum size=1cm, node distance=1.5cm, right=of x2] {$e_1$};

  \path
  %Emission lines
    (x1) edge[out=90,in=-90] node[rectangle,fill=white] {$0.9$} (e11)
    (x1) edge[out=180,in=0] node[rectangle,fill=white] {$0.1$} (e12)
    (x2) edge[out=90,in=-90] node[rectangle,fill=white] {$0.2$} (e21)
    (x2) edge[out=0,in=180] node[rectangle,fill=white] {$0.8$} (e22)

  %Transition lines
    (x1) edge[out=15,in=165] node[rectangle,fill=white] {$0.4$} (x2)
    (x2) edge[out=-165,in=-15] node[rectangle,fill=white] {$0.9$} (x1)

    (x1) edge [out=150,in=120,loop,looseness=15] node[rectangle,fill=white] {$0.6$} (x1)
    (x2) edge [out=60,in=30,loop,looseness=15] node[rectangle,fill=white] {$0.1$} (x2);
\end{tikzpicture}
\end{figure}
\end{frame}

\begin{frame}
\frametitle{Vertebi Example}
\begin{figure}[htbp]
\centering
\begin{tikzpicture}[scale=0.8, every node/.style={transform shape}]
  %Hidden Nodes
  \node[ellipse] (htp1) [draw=black, minimum width=1cm, minimum height=.75cm, right=of dots] {$x_{1}$};
  \node[ellipse] (ht)   [draw=black, minimum width=1cm, minimum height=.75cm, right=of htp1] {$x_{0}$};
  \node[ellipse] (htf1) [draw=black, minimum width=1cm, minimum height=.75cm, right=of ht] {$x_{0}$};
  \node[ellipse] (htfq) [draw=black, minimum width=1cm, minimum height=.75cm, right=of htf1] {$x_{?}$};

  %Emission states
  \node[ellipse] (etp1) [draw=black, minimum width=1cm, minimum height=.75cm, below=of htp1] {$e_{1}$};
  \node[ellipse] (et)   [draw=black, minimum width=1cm, minimum height=.75cm, below=of ht] {$e_{0}$};
  \node[ellipse] (etf1) [draw=black, minimum width=1cm, minimum height=.75cm, below=of htf1] {$e_{0}$};
  \node[ellipse] (etfq) [draw=black, minimum width=1cm, minimum height=.75cm, below=of htfq] {$e_{?}$};

  %Emission lines
  \draw [->, to path={-| (\tikztotarget)}] (htp1) edge[out=-90,in=90] (etp1);
  \draw [->, to path={-| (\tikztotarget)}] (ht) edge[out=-90,in=90] (et);
  \draw [->, to path={-| (\tikztotarget)}] (htf1) edge[out=-90,in=90] (etf1);
  \draw [->, dashed, to path={-| (\tikztotarget)}] (htfq) edge[out=-90,in=90] (etfq);

  %1. order lines
  \draw [->, to path={-| (\tikztotarget)}] (dots) edge[out=0,in=180] (htp1);
  \draw [->, to path={-| (\tikztotarget)}] (htp1) edge[out=0,in=180] (ht);
  \draw [->, to path={-| (\tikztotarget)}] (ht) edge[out=0,in=180] (htf1);
  \draw [->, dashed, to path={-| (\tikztotarget)}] (htf1) edge[out=0,in=180] (htfq);
\end{tikzpicture}
\end{figure}\label{fig:4thMarkovModel}
\end{frame}


\begin{frame}
\frametitle{Vertebi Example}
\begin{figure}[htbp]
\centering
\begin{tikzpicture}[scale=0.8, every node/.style={transform shape}]
  %Hidden Nodes
  \node[ellipse] (htp1) [draw=black, minimum width=1cm, minimum height=.75cm, right=of dots] {$x_{1}$};
  \node[ellipse] (ht)   [draw=black, minimum width=1cm, minimum height=.75cm, right=of htp1] {$x_{0}$};
  \node[ellipse] (htf1) [draw=black, minimum width=1cm, minimum height=.75cm, right=of ht] {$x_{0}$};
  \node[ellipse] (htfq) [draw=black, minimum width=1cm, minimum height=.75cm, right=of htf1] {$x_{0}$};

  %Emission states
  \node[ellipse] (etp1) [draw=black, minimum width=1cm, minimum height=.75cm, below=of htp1] {$e_{1}$};
  \node[ellipse] (et)   [draw=black, minimum width=1cm, minimum height=.75cm, below=of ht] {$e_{0}$};
  \node[ellipse] (etf1) [draw=black, minimum width=1cm, minimum height=.75cm, below=of htf1] {$e_{0}$};
  \node[ellipse] (etfq) [draw=black, minimum width=1cm, minimum height=.75cm, below=of htfq] {$e_{0}$};

  %Emission lines
  \draw [->, to path={-| (\tikztotarget)}] (htp1) edge[out=-90,in=90] (etp1);
  \draw [->, to path={-| (\tikztotarget)}] (ht) edge[out=-90,in=90] (et);
  \draw [->, to path={-| (\tikztotarget)}] (htf1) edge[out=-90,in=90] (etf1);
  \draw [->, to path={-| (\tikztotarget)}] (htfq) edge[out=-90,in=90] (etfq);

  %1. order lines
  \draw [->, to path={-| (\tikztotarget)}] (dots) edge[out=0,in=180] (htp1);
  \draw [->, to path={-| (\tikztotarget)}] (htp1) edge[out=0,in=180] (ht);
  \draw [->, to path={-| (\tikztotarget)}] (ht) edge[out=0,in=180] (htf1);
  \draw [->, to path={-| (\tikztotarget)}] (htf1) edge[out=0,in=180] (htfq);
\end{tikzpicture}
\end{figure}\label{fig:4thMarkovModel}
\end{frame}




\begin{frame}
        \frametitle{Overview}
        \begin{enumerate}
            \item Communication
            \item Normaliser
            \item Sampler
        \end{enumerate}
\end{frame}
\begin{frame}
	\frametitle{Communication, Normaliser and Sampler}
	\begin{itemize}
		\item Communication
\begin{itemize}
\item Receives data from embedded subsystem in binary from
\item Decodes the binary data and sends it to the normaliser
\end{itemize}
\item Normaliser
  \begin{itemize}
\item Receives data from Communication module
\item Normalises the data
\item Sends to Sampler
\end{itemize}

	\end{itemize}

\end{frame}
\begin{frame}
\frametitle{Normaliser}
\begin{itemize}
\item Elbow Method \begin{enumerate}
    \item Start with 2 clusters
    \item Calculate Variance for each clusters
    \item Add new cluster
    \item calculate the variance difference
    \item While Variance difference is below threshold repeat step 3
\end{enumerate}
\item Jenks Natural break optimization
\begin{enumerate}
\item Calculate Sum of Squared Deviation for each cluster (C).
\item Calculate Sum of squared Deviation from mean(M).
\item Move a item from cluster with highest C toward lowest C.
\end{enumerate}
\end{itemize}
\end{frame}
